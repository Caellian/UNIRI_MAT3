\section{Diferencijalni račun}

\subsection{Osnovni teoremi}

\begin{theorembox}[fermatov teorem]
    Neka funkcija $f$ definirana na intervalu $\langle a,b \rangle$ poprima u
    točki $c\in\langle a,b \rangle$ najveću ili najmanju vrijednost. Ako je $f$
    derivabilna u točki $c$, tada je $f'(x)=0$.
\end{theorembox}

% TODO: Graf sa tangentom u maksimumu

\begin{theorembox}[rolleov teorem]
    Neka je funkcija $f$ neprekidna na segmentu $[a,b]$ i derivabilna na $\langle a,b \rangle$.
    Ako je $f(a) = f(b)$, tada postoji barem jedna točka $c\in\langle a,b \rangle$
    u kojoj je $f'(c)=0$.
\end{theorembox}

\begin{theorembox}[teorem o ekstremnim vrijednostima]
    Neka je funkcija $f$ neprekidna na segmentu $[a,b]$. Tada funkcija $f$ na segmentu
    $[a,b]$ postiže minimalnu i maksimalnu vrijednost.
\end{theorembox}

\begin{theorembox}[lagrangeov teorem srednje vrijednosti]
    Neka je funkcija $f$ neprekidna na segmentu $[a,b]$ i derivabilna na $\langle a,b \rangle$.
    Tada postoji barem jedna točka $c\in\langle a,b \rangle$ u kojoj je
    $$
        f'(c) = \frac{f(b)-f(a)}{b-a}
    $$
\end{theorembox}

\subsection{Intervali monotonosti funkcije}


\subsection{Lokalni ekstremi funkcije}


