\section{Diferencijalni račun}

\subsection{Osnovni teoremi}

\begin{theorem}[fermatov teorem]
    Neka funkcija $f$ definirana na intervalu $\langle a,b \rangle$ poprima u
    točki $c\in\langle a,b \rangle$ najveću ili najmanju vrijednost. Ako je $f$
    derivabilna u točki $c$, tada je $f'(c)=0$.

    \begin{center}
    \begin{tikzpicture}
        \begin{axis}[
            tick style={draw=none},
            xtick=\empty,
            ytick=\empty,
            ymin=-15, ymax=25,
            xmin=-1.2, xmax=1.25,
            samples=100,
            smooth,
            width=25em,
            height=15em,
        ]
        
        \addplot [ultra thick, domain=-1.5:1.5] {18*x^3 + 6*x^2 - 10*x + 10} node[pos=0.5, below right] {$f(x)$};
        \addplot [thin, domain=-1.5:1.5] {18*3*x^2 + 12*x - 10} node[pos=0.55, left] {$f'(x)$};
        
        \addplot [cGreen, domain=-1.05:-0.05] coordinates {(-30/54,-3)(-30/54,17.32099)};
        \filldraw [cGreen] (-30/54, 14.32099) circle (2pt) node[black, anchor=south west] {$f(c_1)$};
        \filldraw [black] (-30/54, 0) circle (1pt) node[anchor=north east] {$f'(c_1)$};
        
        \addplot [cRed, domain=-0.16:0.83] coordinates {(18/54,-3)(18/54,11)};
        \filldraw [cRed] (18/54, 8) circle (2pt) node[black, anchor=south east] {$f(c_2)$};
        \filldraw [black] (18/54, 0) circle (1pt) node[anchor=north west] {$f'(c_2)$};
    
        \end{axis}
    \end{tikzpicture}
    \end{center}
\end{theorem}

Pretpostavimo da $f$ ima lokalni maksimum u $x=c$. Tj. $f(c)\geq f(x)$ za svaki
$x$ koji se nalazi blizu $c$.
Uvodimo vrijednost $\pm h$ koja je blizu $0$, za koju
vrijedi $x = c+h$.

Dobivamo izraz oblika:
$$
f(c)\geq f(c+h)
$$

Primjećujemo da je taj izraz sastavljen od dva izraza koji se nalaze u
\fancyeq[definiciji za][derivaciju funkcije]{def:diff}.
Svodimo ga na najednakost s definicijom te pokušavamo svesti $h$ na $0$ kako bi
dobili $c+0=x$:

$$
0 \geq f(c+h) - f(c)\quad /\div h
$$

\begin{multicols}{2}
\begin{equation}
\label{eq:fermat_h_gt_zero}
0\geq\frac{f(c+h)-f(c)}{h}\tag{$h>0$}
\end{equation}

\noindent
Za \fancyeq[slučaj]{eq:fermat_h_gt_zero} možemo vidjeti da sljedeći limes ne može
biti pozitivan jer nije definiran:
$$
    0\geq \lim_{h\to 0^+}\frac{f(c+h)-f(c)}{h}
$$

\newcolumn

\begin{equation}
\label{eq:fermat_h_lt_zero}
0\leq\frac{f(c+h)-f(c)}{h}\tag{$h<0$}
\end{equation}

\noindent
Za \fancyeq[slučaj]{eq:fermat_h_lt_zero} možemo vidjeti da sljedeći limes ne može
biti negativan jer nije definiran:
$$
    0\leq \lim_{h\to 0^-}\frac{f(c+h)-f(c)}{h}
$$

\end{multicols}

Kako bi funkcija $f$ bila derivabilna za $x=c$, oba limesa moraju postojati i
imati jednaku vrijednost za $f'(c)$.

Stoga vrijedi $f'(c)\geq 0 \land f'(c)\leq 0 \implies f'(c)=0$.

Dokaz za minimum je sličan.

\begin{theorem}[rolleov teorem]
    Neka je funkcija $f$ neprekidna na segmentu $[a,b]$ i derivabilna na $\langle a,b \rangle$.
    Ako je $f(a) = f(b)$, tada postoji barem jedna točka $c\in\langle a,b \rangle$
    u kojoj je $f'(c)=0$.

    % TODO: dodati sliku koja prikazuje ideju rolleovog teorema
\end{theorem}

Recimo da je $k=f(a)=f(b)$.

\noindent
Razmatramo sljedeća tri slučaja:
\begin{enumerate}
    \item $f(x) = k, \forall x\in \langle a,b \rangle$.
    \item $\exists x\in \langle a,b \rangle$ takav da $f(x) > k$.
    \item $\exists x\in \langle a,b \rangle$ takav da $f(x) < k$.
\end{enumerate}

\noindent
\textbf{Slučaj 1:} $f(x) = k, \forall x\in \langle a,b \rangle \implies f'(x) = 0, \forall x\in \langle a,b \rangle$.

\bigskip
\noindent
\textbf{Slučaj 2:} Jer je funkcija $f$ neprekidna na zatvorenom intervalu $[a,b]$
i ograničena njime, zbog teorema o ekstremima, ta funkcija ima absolutni maksimum.

Također, jer postoji točka $x\in \langle a,b \rangle$ za koju vrijedi $f(x)>k$,
taj absolutni maksimum je veći od $k$.
Dakle, absolutni maksimum nije ni jedna granična vrijednost.

Posljedično tome, absolutni maksimum se mora pojaviti na unutarnjoj vrijednosti
$c\in\langle a,b \rangle$.

Jer funkcija $f$ ima maksimum na unutarnjoj točki $c$, i $f$ je derivabilna za
$c$, vrijedi $f'(c)=0$.

\bigskip
\noindent
\textbf{Slučaj 3} je analogan slučaju 2, samo je maksimum zamjenjen minimumom.

\newpage
\begin{theorem}[teorem o ekstremima]
    Neka je funkcija $f$ neprekidna na segmentu $[a,b]$. Tada funkcija $f$ na segmentu
    $[a,b]$ postiže minimalnu i maksimalnu vrijednost.

    % TODO: dodati sliku koja prikazuje ideju teorema o ekstremima
\end{theorem}

Jer je funkcija $f$ neprekidna na segmentu $[a,b]$, mora biti ograničena na tom
segmentu prema teoremu o ograničenosti.

\textbf{Pretpostavka:} $\mathbf{M}$ je najmanja gornja međa funkcije $f$.

Ako postoji $c\in[a,b]$ gdje je $f(c)=M$, nema potrebe za daljnim dokazom jer
funkcija $f$ poprima svoj maksimum na intervalu $[a,b]$.

Ako ne postoji takav $c$, onda vrijedi $f(x)<M, \forall x\in [a,b]$.

\textbf{Definiramo:}
$$
g(x) = \frac{1}{M-f(x)}
$$

Funkcija $g(x)>0,\forall x \in [a,b]$ i neprekidna na $[a,b]$, te je samim time
i ograničena na tom intervalu.

S obzirom da je $g$ ograničena na $[a,b]$, mora postojati neki $K>0$ za kojeg
vrijedi $g(x)\leq K, \forall x [a,b]$.

Slijedno tome,
\begin{align*}
\frac{1}{M-f(x)}\leq& K\qquad/\times(M-f(x))\\
1\leq& KM-Kf(x)\\
Kf(x)\leq& KM - 1\qquad/\div K\\
f(x)\leq& M - \frac{1}{K}.\\
\end{align*}

No to nije moguće jer kontradiktira pretpostavci da je $M$ najmanja gornja međa.
Što ostavlja jedino mogučnost da postoji $c\in[a,b]$ gdje je $f(c)=M$.
Drugim riječima, $f$ postiže svoj maksimum na intervalu $[a,b]$.

Dokaz za minimum je sličan maksimumu.

\newpage
\begin{theorem}[lagrangeov teorem srednje vrijednosti]
    \label{thorem:lagrange}

    Neka je funkcija $f$ neprekidna na segmentu $[a,b]$ i derivabilna na $\langle a,b \rangle$.
    Tada postoji barem jedna točka $c\in\langle a,b \rangle$ u kojoj je
    $$
        f'(c) = \frac{f(b)-f(a)}{b-a}
    $$
\end{theorem}

\begin{multicols}{2}

\noindent
$F$ i $G$ su primitivne funkcije od $f$:
$$
\implies \begin{cases}
    F'(x) = f(x),\qquad \forall x\in\langle a,b \rangle\\
    G'(x) = f(x),\qquad \forall x\in\langle a,b \rangle
\end{cases}
$$

\noindent
\textbf{Tvrdnja:}
\begin{align*}
\exists c\in\mathbb{R}\text{ takav da }F(x) = G(x) +& c,\qquad\forall x\in \langle a,b \rangle\\
\Updownarrow\qquad&\\
F(x) - G(x) =& c,\qquad\forall x\in \langle a,b \rangle\\
(F-G)(x) =& c,\qquad\forall x\in \langle a,b \rangle
\end{align*}

\noindent
\textbf{Stavimo:} $H\coloneq F - G \implies H(x)=c,\space\forall x\in \langle a,b \rangle$

\noindent
\textbf{Vrijedi:}

$$
\begin{rcases}
    F'(x) = f(x),\qquad\forall x\in \langle a,b \rangle\\
    G'(x) = f(x),\qquad\forall x\in \langle a,b \rangle
\end{rcases} -
$$
\begin{align*}
    F'(x)-G'(x)&=f(x)-f(x)\\
    (F-G)'(x)&=0,\qquad\forall x\in \langle a,b \rangle\\
    \label{eq:h_is_zero}
    H'(x)&=0,\qquad\forall x\in \langle a,b \rangle\tag{A}
\end{align*}

$$
H'(x)=0\stackrel{?}{\implies}H(x)=c,\qquad\forall x\in \langle a,b \rangle
$$

Neka su $x_1,x_2 \in \langle a,b \rangle,\quad x_1<x_2$.

$$
f \leftrightarrow H,\qquad a\leftrightarrow x_1,\qquad b\leftrightarrow x_2
$$
\newcolumn

\noindent
\textbf{Vrijedi:} $H$ je neprekidna na $[x_1,x_2]\subseteq [a,b]$ i derivabilna
je na $\langle x_1,x_2 \rangle\subseteq\langle a,b \rangle \implies \exists c\in\langle x_1,x_2 \rangle$
u kojoj je
\begin{equation}
    \label{eq:h_is_frac}
    H'(x)=\frac{H(x_2)-H(x_1)}{x_2-x_1}\tag{B}
\end{equation}

Iz \ref{eq:h_is_zero} i \ref{eq:h_is_frac} slijedi:

\begin{align*}
H'(c) = 0 &= \frac{H(x_2)-H(x_1)}{x_2-x_1}\\
H(x_2)-H(x_1) &= 0\\
H(x_1) &= H(x_2),\qquad\forall x_1, x_2\in \langle a,b \rangle
\end{align*}

\noindent
\textbf{Stavimo:}
\begin{align*}
    H(x_1)\coloneq& c,\quad c\in\mathbb{R}\\
    \Downarrow&\\
    H(x_2)=&c,\quad x_2\in\langle a,b \rangle\\
    \Downarrow&\\
    H(x)=&c,\quad x\in\langle a,b \rangle
\end{align*}

\noindent
Dakle, $H(x) = F(x)-G(x)=c,\quad\forall x\in \langle a,b \rangle\implies\exists c\in\mathbb{R}\text{ takav da }F(x)=G(x)+c,\quad \forall x\in \langle a,b \rangle$.

\end{multicols}

\begin{example}
    Neka je $f:[-2,1] \to \mathbb{R}, f(x) = x^2$.
    Odredi točku $c\in\langle -2,1 \rangle$ u kojoj je tangenta na graf zadane
    funkcije paralelna sa sekantom koja prolazi točkama $(-2,4)$ i $(1,1)$.
\end{example}
