\subsection{Derivacije višeg reda}

\begin{definition}[derivacija funkcije višeg reda]
    \textbf{Druga derivacija funkcije $f$} je derivacija funkcije $f'$:

    \begin{equation*}
        f'' = (f')'.
    \end{equation*}

    \textbf{$n$-ta derivacija funkcije $f$} je derivacija funkcije $f^{(n-1)}$:

    \begin{equation*}
        f^{(n)} = \left(f^{(n-1)}\right)'.
    \end{equation*}
\end{definition}

Derivacije višeg reda se zapisuju sa redom derivacije u zagradama, nekad i rimskim
brojevima zbog jasnoće:

$$
f'''''(x) = f^{(5)}(x) = f^{(V)}(x)
$$

\subsection{Diferencijal funkcije}

Diferencijal predstavlja promjenu funkcije u nekoj točki s obzirom na promjenu nezavisne varijable.

\begin{definition}[diferencijal funkcije]
    Izraz $f'(x)\cdot\Delta x$ se zove \textbf{diferencijal funkcije $f$} u točki
    $x$ i označava se sa $dy$ (ili $df(x)$) sukladno oznaci $y=f(x)$ za funkciju $f$.

    \begin{equation*}
        dy = f'(x)\cdot dx
    \end{equation*}
\end{definition}

% potrebno dodati '\usetikzlibrary{decorations.pathreplacing}' u preamble
% za dx i dy. Ili maknuti decoration i staviti latex-latex za obične/ružne
% strijelice.
\begin{center}
\begin{tikzpicture}
    \begin{axis}[
        axis lines=middle,
        xmin=0, xmax=5,
        ymin=0, ymax=25,
        xtick={2,3},
        ytick={10, 20},
        xticklabels={$x$,$x+\Delta x$},
        yticklabels={$f(x)$,$f(x+\Delta x)$},
        domain=0:4,
        samples=100,
        smooth,
        width=30em,
    ]
  
    \addplot [thick, domain=1:4] {2 + 2*x^2};
    \addplot [thin, dashed] {8*x-6};

    \pgfmathsetmacro{\xval}{2}
    \pgfmathsetmacro{\xpval}{3}
    \pgfmathsetmacro{\tangent}{8*\xval - 6}
    \pgfmathsetmacro{\ptangent}{8*\xpval - 6}
    \pgfmathsetmacro{\preal}{2+ 2*\xpval*\xpval}

    % triangle
    \fill [yellow] (\xval,\tangent) coordinate (a) --
          (\xpval,\tangent) coordinate (c) --
          (\xpval,\ptangent) coordinate (b) -- cycle;
    \node [yshift=0.5em, xshift=1.2em] at (\xval, \tangent) {$\varphi$};
          
    % T
    \addplot [mark=*] coordinates {(\xval,\tangent)} node[above left] {\sffamily T};

    % lines
    \addplot [cGreen, thick, dotted] coordinates {(\xval,0) (\xval,\tangent)};
    \addplot [cGreen, thick, dotted] coordinates {(0,\tangent) (\xval,\tangent)};
    \addplot [cBlue, thick, dotted] coordinates {(\xpval,0) (\xpval,\preal)};
    \addplot [cBlue, thick, dotted] coordinates {(0,\preal) (\xpval,\preal)};

    % dx & dy labels
    \draw [decorate, decoration={brace, amplitude=5pt, mirror}, yshift=-2pt] (\xval,\tangent) -- (\xpval,\tangent) node[midway, below=5pt] {$dx$};
    \draw [decorate, decoration={brace, amplitude=5pt, mirror}, xshift=2pt] (\xpval,\tangent) -- (\xpval,\ptangent) node[midway, right=5pt] {$dy$};

    \draw [thick, latex-latex, xshift=32pt] (\xpval,\tangent) -- (\xpval,\preal) node[midway, right=2pt] {$\Delta f(x)$};

    \end{axis}
\end{tikzpicture}
\end{center}

Diferencijal daje linearnu aproksimaciju primjene funkcije $f$ u točki $x$.