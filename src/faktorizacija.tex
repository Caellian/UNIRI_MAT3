\section{Faktorizacija polinoma}

Za faktorizaciju složenijih polinoma su korisne formule za jednostavnije
faktorizacije:

\begin{center}
    \begin{tabular}{r|rcl}
        Ime & Oblik && Faktorizacija \\
        \hline
        Razlika kvadrata & $a^2 - b^2$ &=& $(a+b)(a-b)$ \\
        Zbroj ili razlika kubova & $a^3 \pm b^3$ &=& $(a \pm b)(a^2 \mp ab + b^2)$ \\
        Kvadrat zbroja ili razlike & $a^2 \pm 2ab + b^2$ &=& $(a \pm b)^2$ \\
        Kub zbroja ili razlike & $a^3 \pm 3a^2b + 3ab^2 \pm b^3$ &=& $(a \pm b)^3$ \\
        \hline
        \hline
        Zbroj kvadrata & $a^2 + b^2$ &=& $(a+b\,i)(a-b\,i)$ \\
    \end{tabular}
\end{center}

\noindent
Polinom $Q_m(x)$ stupnja $m$ ima točno $m$ korijena (samo jedan korijen ako je
$m=0$), kada ubrajamo i ponavljanje. Ako je $x_i$ korijen polinoma $Q_m(x)$ tada
je $Q_m(x)$ djeljiv s $(x-x_i)$, odnosno $Q_m(x) = (x-x_i)Q_{m-1}(x)$.

\textbf{Faktorizacija polinoma} $Q_m=a_mx^m+\dots+a_1x+a_0$ zapis je u obliku
$$
Q_m(x) = a_m(x-x_1)^{n_1}\dots(x-x_n)^{n_k}\cdot(x^2+p_1x+q_1)^{m_1}\dots(x^2+p_1x+q_1)^{m_l},
$$

\begin{itemize}
    \item Ako je $\tilde{x}$ realni korijen polinoma $Q_m$ tada je
    $$
    Q_m(x) = (x-\tilde{x})Q_{m-1}(x),
    $$
    pa za nalaženje ostalih faktora treba faktorizirati polinom $Q_{m-1}$
    stupnja $m-1$.
    \item Ako su $z$ i $\tilde{z}$ par kompleksno konjugiranih korijena polinoma
    $Q_m$ tada je
    $$
    Q_m(x) = (x^2+px+q)Q_{m-2}(x),
    $$
    pri čemu je $(x^2+px+q) = (x-z)(x-\tilde{z})$ pa za nalaženje ostalih
    faktora treba faktorizirati polinom $Q_{m-2}$ stupnja $m-2$.
\end{itemize}

\begin{example}
    Faktorizirajte polinom:
    $Q_3(x) = x^3-5x^2+9x-5$
\end{example}

Konstanta uz vodeći koeficijent je $6$, a konstanta uz slobodni član je $-2$.
Njegovi korijeni su onda kombinacije svih mogućih dijelitelja tih konstanta u
obliku $\pm\frac{p}{q}$, gdje je $p$ dijelitelj konstante uz slobodni član, a
$q$ dijelitelj konstante uz vodeći koeficijent.

U danom primjeru to su:
\begin{gather*}
p\in\{\pm1, \pm3\}\\
q\in\{\pm1, \pm5\}
\end{gather*}

Tako da su svi mogući korijeni polinoma $Q_3(x)$:
\begin{gather*}
\pm 1/1, \pm 1/5,\\
\pm 3/1, \pm 3/5.
\end{gather*}

Uvrštavanjem u polinom redom, nalazimo korijen polinoma za kojeg je vrijednost
polinoma jednaka $0$. U ovom slučaju to je $x=1/1$ tj. $1$.

Tu se radi o realnom korijenu pa je faktorizacija polinoma $Q_3(x)$:
$$
Q_3(x) = (x-1)Q_2(x) = (x-1)(5x^2+9x-5).
$$

Određujemo $Q_2$

\begin{example}
    Faktorizirajte polinom: $Q_3(x) = x^3-3x^2+4x-2$
\end{example}

Znamo da postoji točno $3$ korijena polinoma $Q_3(x)$ jer je broj korijena
jednak stupnju polinoma.


konstanta uz vodeći koeficijent je 6,
a konstanta uz slobodni član je $-2$. Dobivamo moguće korijene:

\begin{gather*}
p\in\{\pm1, \pm3\},\\
q\in\{\pm1, \pm2\},\\
\pm 1/1, \pm 1/2, \pm 3/1, \pm 3/2.\\
\end{gather*}

Uvrštavanjem u polinom redom, nalazimo korijen polinoma za kojeg je vrijednost
polinoma jednaka $0$. U ovom slučaju to je $x=1/1$ tj. $1$.


\begin{example}
    Faktorizirajte polinom:
    \begin{enumerate}
        \item $Q_2(x) = x^2-10x+5$,
        \item $Q_3(x) = x^3-2x^2+x-2$,
        \item $Q_3(x) = x^3+3x-9$
    \end{enumerate}
\end{example}
