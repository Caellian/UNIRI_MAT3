\subsection{Deriviranje implicitno zadanih funkcija}

Implicitno zadanu funkciju $F(x,y) = 0$ se derivira tako da se izraze koji sadrže
zavisnu varijablu $y=y(x)$ derivira koristeći pravilo za deriviranje kompozicije
funkcija.

Ne smijemo tretirati $y$ kao konstantu jer ako izrazimo $y$ kao funkciju od $x$
možemo primjetiti da su im vrijednosti međusobno zavisne.

\begin{multicols}{2}
\begin{example}[deriviranje implicitno zadane funkcije]
    Deriviraj implicitno zadanu funkciju: $\displaystyle x^2 + y^2 = 1$
\end{example}

Deriviramo obje strane zadanog izraza:
\smallskip

\begin{tabular}{cl}
    \vspace*{11pt}$\displaystyle \frac{d}{dx}\left[x^2 + y^2\right] = \frac{d}{dx}[1]$\\
    \vspace*{11pt}$\displaystyle \frac{d}{dx}\left[x^2\right] + \frac{d}{dx}\left[y^2\right] = 0$\\
    \vspace*{11pt} $y^2$ tretiramo kao $y(x)$: $\displaystyle \frac{d}{dx}[(y(x))^2] = 2y\cdot\frac{dy}{dx}$\\
    \vspace*{11pt}$\displaystyle 2x + \frac{d(y^2)}{dy}\cdot\frac{dy}{dx} = 0$\\
    \vspace*{11pt}$\displaystyle 2x + 2y\cdot\frac{dy}{dx} = 0$\\
    \vspace*{11pt}$\displaystyle\phantom{\quad / \div 2y} 2y\cdot\frac{dy}{dx} = -2x\quad / \div 2y$ \\
    \vspace*{11pt}$\displaystyle\frac{dy}{dx} = -\frac{x}{y}$
\end{tabular}

\vspace*{\stretch{1}}
\columnbreak

\begin{example}[nagib implicitno zadane funkcije]
    Odredite nagib koji poprima implicitno zadana funkcija $x^2+(y-x)^3=28$ kada je $x=1$.
\end{example}

Kako bismo dobili $y$, uvrštavamo $x=1$ u zadanu funkciju:
\begin{gather*}
    1+(y-1)^3 = 28\\
    (y-1)^3 = 27\\
    y-1 = 3\\
    y = 4
\end{gather*}

Dakle zanima nas nagib funkcije za točku $(1,4)$.

\begin{gather*}
    \frac{d}{dx}\left[x^2+(y-x)^3\right]=\frac{d}{dx}\left[28\right]\\
    2x+3(y-x)^2\left(\frac{dy}{dx} - 1\right) = 0\\
    2x+3(y-x)^2\frac{dy}{dx} - 3(y-x)^2 = 0\\
    3(y-x)^2\frac{dy}{dx} = 3(y-x)^2 - 2x\\
    \frac{dy}{dx} = \frac{3(y-x)^2 - 2x}{3(y-x)^2} = 1 - \frac{2x}{3(y-x)^2}
\end{gather*}

Uvrštavanjem $x=1$ i $y=4$ u dobiveni izraz, dobivamo nagib funkcije $(\frac{dy}{dx})$ u točki $(1,4)$:

$$
\frac{dy}{dx} = 1 - \frac{2\cdot 1}{3(4-1)^2} = 1 - \frac{2}{3\cdot 9} = 1 - \frac{2}{27} = \frac{25}{27}
$$

\end{multicols}

\newpage

\subsection{Logaritamsko deriviranje}

Logaritamsko deriviranje se koristi za deriviranje funkcija oblika: $\displaystyle
y=h(x)=f(x)^{g(x)}
$

U onim točkama u kojima derivacija postoji vrijedi:

$$
\left(f(x)^{g(x)}\right)' = f(x)^{g(x)}\left(g'(x)\ln(f(x))+g(x)\frac{f'(x)}{f(x)}\right).
$$

Postupak kojim se dolazi do derivacije sastoji se od tri koraka:
\begin{itemize}
    \item logaritmiranje obje strane,
    \item deriviranje obje strane, pri čemu se $y$ derivira kao složena funkcija (kompozicija),
    \item sređivanje dobivene jednakosti.
\end{itemize}

\begin{multicols}{2}
\begin{example}[logaritamsko deriviranje]
    Deriviraj funkciju $y=x^{x^2}$.
\end{example}

\begin{align*}
    y=x^{x^2} \xrightarrow{\ln} \ln y=& \underbrace{\ln \left(x^{x^2}\right)}_{x^2 \ln x}\\
    \frac{d}{dx}[\ln y] =& \frac{d}{dx} \left[ x^2 \ln x \right]\\
    \frac{dy}{dx}\cdot\frac{1}{y}=& 2x \ln x + x^{\cancel{2}}\cdot\frac{1}{\cancel{x}}\\
    \frac{dy}{dx} = y (2x\ln x + x) =& \boxed{x^{x^2} (2x\ln x + x)}
\end{align*}

\columnbreak

\begin{example}[logaritamsko deriviranje]
    Deriviraj funkciju $y=(\ln x)^{\sin x}$.
\end{example}

\begin{align*}
    \ln x = &\sin x \ln (\ln x)\\
    \frac{dy}{dx}\cdot\frac{1}{y} =& \cos x \ln (\ln x) + \sin x \cdot \frac{1}{\ln x} \cdot \frac{1}{x}\\
    \frac{dy}{dx} =& y (\cos x \ln (\ln x) + \frac{\sin x}{x \ln x})\\
    \frac{dy}{dx} =& \boxed{(\ln x)^{\sin x}\left[\cos x \ln (\ln x) + \frac{\sin x}{x \ln x}\right]}
\end{align*}

\end{multicols}


\subsection{Deriviranje parametarski zadane funkcije}

Kada je funkcija $y=y(x)$ parametarski zadana sa: $\displaystyle
\begin{cases}
    x = x(t)\\
    y = y(t)
\end{cases},
$ tada je:

\begin{center}
    \begin{tabular}{ccccc}
    $y'$&$=$&$\displaystyle\frac{\dot{y}}{\dot{x}}$&$=$&$\displaystyle\frac{\frac{dy}{dt}}{\frac{dx}{dt}}$,\\
    $y''$&$=$&$\displaystyle\frac{\dot{y'}}{\dot{x}}$,\\
    &$\vdots$&\\
    $y^{(n)}$&$=$&$\displaystyle\frac{\dot{y^{(n-1)}}}{\dot{x}}$.
    \end{tabular}
\end{center}

\begin{example}[deriviranje parametarski zadane funkcije]
    Odredi $y'$ i $y''$ za sljedeće funkcije:

    \begin{enumerate}
        \item $x=t,\quad y=1-t,\quad t\in\mathbb{R}$
        \item $x=t,\quad y=t^2,\quad t\in\mathbb{R}$
        \item $x=3\cos t,\quad y=\sin t,\quad t\in[0, 2\pi]$
    \end{enumerate}
\end{example}

