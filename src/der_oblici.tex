\section{Deriviranje implicitno zadanih funkcija}

Implicitno zadanu funkciju $F(x,y) = 0$ se derivira tako da se izraze koji sadrže
zavisnu varijablu $y=y(x)$ derivira koristeći pravilo za deriviranje kompozicije
funkcija.

\begin{examplebox}[deriviranje implicitno zadane funkcije]
    Deriviraj implicitno zadane funkcije:

    \begin{enumerate}
        \item $xy-1=0$
        \item $y^2-x=0$
        \item $xe^y+y=0$
        \item $y^2-xy+x\ln y = 1$
    \end{enumerate}
\end{examplebox}

\section{Logritamsko deriviranje}

Logaritamsko deriviranje se koristi za deriviranje funkcija oblika: $\displaystyle
y=h(x)=f(x)^{g(x)}
$

U onim točkama u kojima derivacija postoji vrijedi:

$$
\left(f(x)^{g(x)}\right)' = f(x)^{g(x)}\left(g'(x)\ln(f(x))+g(x)\frac{f'(x)}{f(x)}\right).
$$

Postupak kojim se dolazi do derivacije sastoji se od tri koraka:
\begin{itemize}
    \item logaritmiranje obje strane,
    \item deriviranje obje strane, pri čemu se $y$ derivira kao složena funkcija (kompozicija),
    \item sređivanje dobivene jednakosti.
\end{itemize}

\begin{examplebox}[logaritamsko deriviranje]
    Deriviraj funkcije:

    \begin{enumerate}
        \item $y=x^{x^2}$
        \item $y=(\ln x)^{\sin x}$
    \end{enumerate}
\end{examplebox}

\section{Deriviranje parametarski zadane funkcije}

Kada je funkcija $y=y(x)$ parametarski zadana sa: $\displaystyle
\begin{cases}
    x = x(t)\\
    y = y(t)
\end{cases},
$ tada je:

\begin{align*}
y'&=\frac{\dot{y}}{\dot{x}},\\
y''&=\frac{\dot{y'}}{\dot{x}},\\
\vdots&\\
y^{(n)}&=\frac{\dot{y^{(n-1)}}}{\dot{x}}.
\end{align*}

\begin{examplebox}[parametarski zadane funkcije]
    Odredi $y'$ i $y''$ za sljedeće funkcije:

    \begin{enumerate}
        \item $x=t,\quad y=1-t,\quad t\in\mathbb{R}$
        \item $x=t,\quad y=t^2,\quad t\in\mathbb{R}$
        \item $x=3\cos t,\quad y=\sin t,\quad t\in[0, 2\pi]$
    \end{enumerate}
\end{examplebox}