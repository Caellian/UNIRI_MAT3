\section{Derivacije funkcije jedne realne varijable}

Derivacija funkcije $f'(x_0)$ opisuje tok funkcije $f$ u točki $x_0$. Kada je
pozitivna, vrijednost funkcije rase, kada je negativna, vrijednost funkcije pada.

\begin{definitionbox}[derivacija funkcije]
    Ako za funkciju $f:\langle a,b\rangle \to \mathbb{R}$ u točki $x_0\in\langle a,b\rangle$ postoji:

    \begin{equation}
        \label{eq:diff}
        \tag{derivacija funkcije}
        \lim_{x\to x_0} \frac{f(x) - f(x_0)}{x-x_0},
    \end{equation}

    tada taj limes označavamo s $f'(x_0)$, zovemo ga \textbf{derivacijom funkcije $f$ u točki $x_0$}
    te kažemo da je $f$ \textbf{derivabilna u $x_0$}
\end{definitionbox}

Druge oznake:

$$
f'(x_0) = \frac{df}{dx}(x_0) = f(x_0)
$$

\begin{propositionbox}[veza derivabilnosti i neprekidnosti]
    Ako je funkcija $f$ derivabilna u točki $x_0$, tada je $f$ neprekidna u točki $x_0$.
\end{propositionbox}

Kako bi funkcija $f$ bila neprekidna u točki $x_0$, ona mora:
\begin{itemize}
    \item biti definirana u točki $x_0$,
    \item $\lim_{x\to x_0}f(x)$ mora postojati, i
    \item $\lim_{x\to x_0}f(x) = f(x_0)$
\end{itemize}

Jer je funkcija $f(x)$ derivabilna u $x=x_0$, ona je \textbf{definirana u točki $x_0$} i znamo da je \fancyeq[formula][derivacije funkcije]{eq:diff} valjana:
$$
f'(x) = \lim_{x\to x_0} \frac{f(x) - f(x_0)}{x-x_0}.
$$

Ako pretpostavimo da $x\neq x_0$ možemo zapisati:
$$
f(x) - f(x_0) = \frac{f(x)-f(x_0)}{x-x_0}(x-x_0).
$$

Uporabom pravila o zbroju limesa iz toga možemo izvesti:

\begin{align*}
\lim_{x\to x_0}(f(x) - f(x_0)) =& \lim_{x\to x_0}\left[\frac{f(x)-f(x_0)}{x-x_0}(x-x_0)\right]\\
=&\lim_{x\to x_0}\frac{f(x)-f(x_0)}{x-x_0}\lim_{x\to x_0}(x-x_0).
\end{align*}

Primjećujemo da je prvi član jednak $f'(x_0)$ (prema \fancyeq[definiciji za][deriviranje]{eq:diff}), a drugi je jednak 0, tako da vrijedi:

\begin{equation}
    \label{eq:diff_zero_sized}
    \lim_{x\to x_0}(f(x)-f(x_0)) = f'(x_0)\cdot0 = 0.
\end{equation}

Dodajemo nulu s desne strane $\lim_{x\to x_0}f(x)$ te ponovo primjenimo pravilo o zbroju limesa
te uvrštavamo u \fancyeq[izraz]{eq:diff_zero_sized}, čime dobivamo:
\begin{align*}
\lim_{x\to x_0}f(x) =& \lim_{x\to x_0}[f(x)+\overbrace{f(x_0)-f(x_0)}^0]\\
=& \lim_{x\to x_0}f(x_0) + \lim_{x\to x_0}[f(x)-f(x_0)]\\
=& \lim_{x\to x_0}f(x_0) + 0,\\
\lim_{x\to x_0}f(x) =& f(x_0).
\end{align*}

Time smo dokazali zadnja dva preduvijeta te i samu propoziciju.
