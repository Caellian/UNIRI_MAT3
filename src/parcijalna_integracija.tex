\subsection{Parcijalna integracija}

\textbf{Metoda parcijalne integracije} ili \textbf{uv-supstitucija} je postupak
kojim se integrira produkt dvije funkcije. Izvodi se iz \fancyeq[pravila
za][derivaciju produkta funkcija]{eq:diff_prod}. Primjenjuje se kada želimo
integrirati produkt dvije funkcije.


Formula za primjenu parcijalne integracije je:

$$
\int u\,dv = uv - \int v\,du\,.
$$

Metoda parcijalne integracije se provodi tako što se:
\begin{enumerate}
    \item odaberu izrazi $u$ i $dv$ takve da je za njih podintegralna funkcija
    jednaka $u\,dv$,
    \item računaju $du$ i $v$,
    \item uvrštavamo nazad u $\displaystyle uv - \int v\,du$, te
    \item rješava preostali podintegralni izraz oblika $v\,du$.
\end{enumerate}

$u$ i $dv$ biramo na tako da $v\,du$ bude što lakše integrirati, tj. da se što
više "pojednostavi" podintegralni izraz. Ako je dobiveni podintegralni izraz
jednako složen kao i početni, onda su $u$ i $dv$ loše odabrani.

Ako je $P(x)$ polinom $n$-tog stupnja, a $a$ realan broj, onda parcijalnom
integracijom rješavamo integrale sljedećeg tipa:

$$
\begin{array}{lll}
    \displaystyle \int P(x)\sin ax\,dx & u=P(x) & dv=\sin ax\,dx,\\
    \displaystyle \int P(x)\cos ax\,dx & u=P(x) & dv=\cos ax\,dx,\\
    \displaystyle \int P(x)e^{ax}dx & u=P(x) & dv=e^{ax}dx,\\
    \displaystyle \int P(x)\ln ax\,dx & u=\ln ax\,dx & dv=P(x),\\
    \displaystyle \int P(x)\arctg ax\,dx & u=\arctg ax\,dx & dv=P(x),\\
    \displaystyle \int P(x)\arcsin ax\,dx & u=\arcsin ax\,dx & dv=P(x).
\end{array}
$$

U prva je tri slučaja broj potrebnih parcijalnih integracija jednak stupnju
polinoma $P(x)$ jer se u svakom koraku stupanj polinoma u podintegralnoj
funkciji snižava za jedan.

\begin{example}
    Izračunajte integrale:
    \begin{enumerate}
        \item $\displaystyle \int (x^2+3)e^xdx$
        \item $\displaystyle \int (2x+1)\ln(2x)dx$
        \item $\displaystyle \int (2x-3)\sin(5x) dx$
        \item $\displaystyle \int x^2\sin(x)dx$
    \end{enumerate}
\end{example}

\subsection{Racionalne funkcije}

\textbf{Racionalna funkcija} je funkcija oblika
$$
R(x) = \frac{P_n(x)}{Q_m(x)},
$$
pri čemu su $P_n(x)$ i $Q_m(x)$ polinomi stupnja $n$, odnostno $m$.

Ukoliko je $n < m$, kažemo da je racionalna funkcija \textbf{prava}.

Ako je $n \geq m$, polinom u brojniku $P_n(x)$ dijelimo polinomom u nazivniku
$Q_m(x)$ te koristeći rezultat dijeljenja $S_{n-m}$ i ostatak dijeljenja $T_k$,
$k<m$ racionalnu funkciju $R$ rastavljamo na polinom i pravu racionalnu
funkciju:
$$
R(x) = S_{n-m}(x) + \frac{T_k(x)}{Q_m(x)}.
$$

\begin{example}
    Funkciju $R(x)=\frac{x^3+2x^2-2x+4}{x+3}$ rastavite na polinom i pravu
    racionalnu funkciju.
\end{example}

\begin{multicols}{2}
\begin{alignat*}{9}
                &&& x^2 &-& x &+& 1\\
                \cline{2-9}
x+3 \quad|\quad &x^3 &+& 2x^2 &-& 2x &+& 4 \\
                -(&x^3 &+& 3x^2)\\
                \cline{2-9}
                && -&x^2 &-&2x &+& 4 \\
                && -(-&x^2 &-& 3x) \\
                \cline{2-9}
                &&&&&x &-& 4\\
                &&&&-(&x &+& 3)\\
                \cline{2-9}
                &&&&&&&1
\end{alignat*}
\columnbreak
\vspace*{\stretch{1.4}}
\begin{align*}
    \frac{x^3}{x} = \boxed{x^2}&&(x+3)\cdot x^2 = \boxed{x^3 + 3x^2}&\\\\
    \frac{-x^2}{x} = \boxed{-x}&&(x+3)\cdot -x = \boxed{-x^2 - 3x}&\\\\
    \frac{x}{x} = \boxed{1}&&(x+3)\cdot 1 = \boxed{x+3}&\\
\end{align*}
\vspace*{\stretch{0.6}}
\end{multicols}

\subsection{Faktorizacija polinoma}

Za faktorizaciju složenijih polinoma su korisne formule za jednostavnije
faktorizacije:

\begin{center}
    \begin{tabular}{r|rcl}
        Ime & Oblik && Faktorizacija \\
        \hline
        Razlika kvadrata & $a^2 - b^2$ &=& $(a+b)(a-b)$ \\
        Zbroj ili razlika kubova & $a^3 \pm b^3$ &=& $(a \pm b)(a^2 \mp ab + b^2)$ \\
        Kvadrat zbroja ili razlike & $a^2 \pm 2ab + b^2$ &=& $(a \pm b)^2$ \\
        Kub zbroja ili razlike & $a^3 \pm 3a^2b + 3ab^2 \pm b^3$ &=& $(a \pm b)^3$ \\
        \hline
        \hline
        Zbroj kvadrata & $a^2 + b^2$ &=& $(a+b\,i)(a-b\,i)$ \\
    \end{tabular}
\end{center}

\par
Polinom $Q_m(x)$ stupnja $m$ ima točno $m$ korijena (samo jedan korijen ako je
$m=0$), kada ubrajamo i ponavljanje. Ako je $x_i$ korijen polinoma $Q_m(x)$ tada
je $Q_m(x)$ djeljiv s $(x-x_i)$, odnosno $Q_m(x) = (x-x_i)Q_{m-1}(x)$.

\textbf{Faktorizacija polinoma} $Q_m=a_mx^m+\dots+a_1x+a_0$ zapis je u obliku
$$
Q_m(x) = a_m(x-x_1)^{n_1}\dots(x-x_n)^{n_k}\cdot(x^2+p_1x+q_1)^{m_1}\dots(x^2+p_1x+q_1)^{m_l},
$$

\begin{itemize}
    \item Ako je $\tilde{x}$ realni korijen polinoma $Q_m$ tada je
    $$
    Q_m(x) = (x-\tilde{x})Q_{m-1}(x),
    $$
    pa za nalaženje ostalih faktora treba faktorizirati polinom $Q_{m-1}$
    stupanja $m-1$.
    \item Ako su $z$ i $\tilde{z}$ par kompleksno konjugiranih korijena polinoma
    $Q_m$ tada je
    $$
    Q_m(x) = (x^2+px+q)Q_{m-2}(x),
    $$
    pri čemu je $(x^2+px+q) = (x-z)(x-\tilde{z})$ pa za nalaženje ostalih
    faktora treba faktorizirati polinom $Q_{m-2}$ stupnja $m-2$.
\end{itemize}

\begin{example}
    Faktorizirajte polinom:
    $Q_3(x) = x^3-5x^2+9x-5$
\end{example}

Konstanta uz vodeći koeficijent je $6$, a konstanta uz slobodni član je $-2$.
Njegovi korijeni su onda kombinacije svih mogućih dijelitelja tih konstanta u
obliku $\pm\frac{p}{q}$, gdje je $p$ dijelitelj konstante uz slobodni član, a
$q$ dijelitelj konstante uz vodeći koeficijent.

U danom primjeru to su:
\begin{gather*}
p\in\{\pm1, \pm3\}\\
q\in\{\pm1, \pm5\}
\end{gather*}

Tako da su svi mogući korijeni polinoma $Q_3(x)$:
\begin{gather*}
\pm 1/1, \pm 1/5,\\
\pm 3/1, \pm 3/5.
\end{gather*}

Uvrštavanjem u polinom redom, nalazimo korijen polinoma za kojeg je vrijednost
polinoma jednaka $0$. U ovom slučaju to je $x=1/1$ tj. $1$.

Tu se radi o realnom korijenu pa je faktorizacija polinoma $Q_3(x)$:
$$
Q_3(x) = (x-1)Q_2(x).
$$

\textbf{Primjer kompleksnog korijena:}

Ako je zadan polinom $Q_3(x) = x^3-3x^2+4x-2$. Znamo da postoji točno $3$
korijena polinoma $Q_3(x)$ jer je broj korijena jednak stupnju polinoma.


konstanta uz vodeći koeficijent je 6,
a konstanta uz slobodni član je $-2$. Dobivamo moguće korijene:

\begin{gather*}
p\in\{\pm1, \pm3\},\\
q\in\{\pm1, \pm2\},\\
\pm 1/1, \pm 1/2, \pm 3/1, \pm 3/2.\\
\end{gather*}

Uvrštavanjem u polinom redom, nalazimo korijen polinoma za kojeg je vrijednost
polinoma jednaka $0$. U ovom slučaju to je $x=1/1$ tj. $1$.


\begin{example}
    Faktorizirajte polinom:
    \begin{enumerate}
        \item $Q_2(x) = x^2-10x+5$,
        \item $Q_3(x) = x^3-3x^2+4x$,
        \item $Q_3(x) = x^3-2x^2+x-2$,
        \item $Q_3(x) = x^3+3x-9$
    \end{enumerate}
\end{example}

\subsection{Parcijalni razlomci}

Pravu racionalnu funkciju možemo rastaviti na \textbf{parcijalne razlomke} ako
nam je poznata faktorizacija polinoma $Q_m$ u nazivniku.

Prilikom rastave na parcijalne razlomke vrijedi
\begin{itemize}
    \item svkom linearnom faktoru $(x-x_i)^{x_i}$ polinoma iz nazivnika odgovra
    izraz
    $$
      \frac{A_{i1}}{x-x_i} + \frac{A_{i2}}{(x-x_i)^2} + \dots + \frac{A_{in_i}}{(x-x_i)^{n_i}},
    $$
    \item svakom kvadratnom faktoru $(x^2+p_jx+q_j)^{m_j}$ polinoma iz nazivnika
    odgovara izraz
    $$
      \frac{B_{j1}x+C_{j1}}{x^2+p_jx+q_j} + \frac{B_{j2}x+C_{j2}}{(x^2+p_jx+q_j)^2} + \dots + \frac{B_{jm_j}x+C_{jm_j}}{(x^2+p_jx+q_j)^{m_j}}.
    $$
\end{itemize}

Nepoznate koeficijenate $A_{il}$, $B_{jk}$ i $C_{jk}$ nalazimo iz same
jednakosti rastava (množenjem s nazivnikom i izjednačavanjem koeficijenata uz
iste postencije od $x$).

\begin{example}
    Rastavite na parcijalne razlomke funkcije:
    \begin{enumerate}
        \item $R(x)=\frac{x+2}{x^2-1}$,
        \item $R(x)=\frac{2x^2+x+2}{x^3+x}$,
        \item $R(x)=\frac{1}{x^3+2x^2+x}$,
        \item $R(x)=\frac{x+1}{x^3-x^2}$.
    \end{enumerate}
\end{example}

\subsection{Integriranje parcijalnih razlomaka}

Integrale parcijalnih razlomaka svodimo na tablične integrale na sljedeći način:

\begin{enumerate}
    \item integral $\int \frac{dx}{(x-a)^n}$ supstitucijom $t=x-a$,
    \item integral $\int \frac{2x+p}{(x^2+px+q)^n}$ supstitucijom $t=x^2+px+q$.
\end{enumerate}

\begin{example}
    Izračunajte integral $\int R(x) dx$:
    \begin{enumerate}
        \item za racionalne funkcije $R(x)$ iz prethodnih primjera,
        \item za $R(x) = \frac{x^5+x^4-8}{x^3-4x}$
    \end{enumerate}
\end{example}
