\subsection{Parcijalna integracija}

\textbf{Metoda parcijalne integracije} ili \textbf{uv-supstitucija} je postupak
kojim se integrira produkt dvije funkcije. Izvodi se iz \fancyeq[pravila
za][derivaciju produkta funkcija]{eq:diff_prod}. Primjenjuje se kada želimo
integrirati produkt dvije funkcije.


Formula za primjenu parcijalne integracije je:

$$
\int u\,dv = uv - \int v\,du\,.
$$

Metoda parcijalne integracije se provodi tako što se:
\begin{enumerate}
    \item odaberu izrazi $u$ i $dv$ takve da je za njih podintegralna funkcija
    jednaka $u\,dv$,
    \item računaju $du$ i $v$,
    \item uvrštavamo nazad u $\displaystyle uv - \int v\,du$, te
    \item rješava preostali podintegralni izraz oblika $v\,du$.
\end{enumerate}

$u$ i $dv$ biramo na tako da $v\,du$ bude što lakše integrirati, tj. da se što
više "pojednostavi" podintegralni izraz. Ako je dobiveni podintegralni izraz
jednako složen kao i početni, onda su $u$ i $dv$ loše odabrani.

Ako je $P(x)$ polinom $n$-tog stupnja, a $a$ realan broj, onda parcijalnom
integracijom rješavamo integrale sljedećeg tipa:

$$
\begin{array}{lll}
    \displaystyle \int P(x)\sin ax\,dx & u=P(x) & dv=\sin ax\,dx,\\
    \displaystyle \int P(x)\cos ax\,dx & u=P(x) & dv=\cos ax\,dx,\\
    \displaystyle \int P(x)e^{ax}dx & u=P(x) & dv=e^{ax}dx,\\
    \displaystyle \int P(x)\ln ax\,dx & u=\ln ax\,dx & dv=P(x),\\
    \displaystyle \int P(x)\arctg ax\,dx & u=\arctg ax\,dx & dv=P(x),\\
    \displaystyle \int P(x)\arcsin ax\,dx & u=\arcsin ax\,dx & dv=P(x).
\end{array}
$$

U prva je tri slučaja broj potrebnih parcijalnih integracija jednak stupnju
polinoma $P(x)$ jer se u svakom koraku stupanj polinoma u podintegralnoj
funkciji snižava za jedan.

\begin{example}
    Izračunajte integrale:
    \begin{enumerate}
        \item $\displaystyle \int (x^2+3)e^xdx$
        \item $\displaystyle \int (2x+1)\ln(2x)dx$
        \item $\displaystyle \int (2x-3)\sin(5x) dx$
        \item $\displaystyle \int x^2\sin(x)dx$
    \end{enumerate}
\end{example}

\subsection{Racionalne funkcije}

\textbf{Racionalna funkcija} je funkcija oblika
$$
R(x) = \frac{P_n(x)}{Q_m(x)},
$$
pri čemu su $P_n(x)$ i $Q_m(x)$ polinomi stupnja $n$, odnostno $m$.

Ukoliko je $n < m$, kažemo da je racionalna funkcija \textbf{prava}.

Ako je $n \geq m$, polinom u brojniku $P_n(x)$ dijelimo polinomom u nazivniku
$Q_m(x)$ te koristeći rezultat dijeljenja $S_{n-m}$ i ostatak dijeljenja $T_k$,
$k<m$ racionalnu funkciju $R$ rastavljamo na polinom i pravu racionalnu
funkciju:
$$
R(x) = S_{n-m}(x) + \frac{T_k(x)}{Q_m(x)}.
$$

\begin{example}
    Funkciju $R(x)=\frac{x^3+2x^2-2x+4}{x+3}$ rastavite na polinom i pravu
    racionalnu funkciju.
\end{example}

\begin{multicols}{2}
\begin{alignat*}{9}
                &&& x^2 &-& x &+& 1\\
                \cline{2-9}
x+3 \quad|\quad &x^3 &+& 2x^2 &-& 2x &+& 4 \\
                -(&x^3 &+& 3x^2)\\
                \cline{2-9}
                && -&x^2 &-&2x &+& 4 \\
                && -(-&x^2 &-& 3x) \\
                \cline{2-9}
                &&&&&x &-& 4\\
                &&&&-(&x &+& 3)\\
                \cline{2-9}
                &&&&&&&1
\end{alignat*}
\columnbreak
\vspace*{\stretch{1.4}}
\begin{align*}
    \frac{x^3}{x} = \boxed{x^2}&&(x+3)\cdot x^2 = \boxed{x^3 + 3x^2}&\\\\
    \frac{-x^2}{x} = \boxed{-x}&&(x+3)\cdot -x = \boxed{-x^2 - 3x}&\\\\
    \frac{x}{x} = \boxed{1}&&(x+3)\cdot 1 = \boxed{x+3}&\\
\end{align*}
\vspace*{\stretch{0.6}}
\end{multicols}
