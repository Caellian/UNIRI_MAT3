\section{Pravila deriviranja}

\begin{multicols}{2}

\begin{propositionbox}[pravilo o zbroju funkcija]
    Derivacija zbroja je jednaka zbroju derivacija:

    $$
        (f(x)+g(x))' = f'(x) + g'(x)
    $$
\end{propositionbox}

Raspisujemo zbroj funkcija kroz \fancyeq[definiciju za][derivaciju funkcije]{eq:diff},
uređujemo nazivnik i primjenjujemo pravilo o zbroju limesa:

\begin{align*}
    &(f(x)+g(x))'=\\
    &=\lim_{h\to 0}\frac{f(x+h)+g(x+h)-(f(x)+g(x))}{h}\\
    &=\lim_{h\to 0}\frac{f(x+h)-f(x)+g(x+h)-g(x)}{h}\\
    &=\lim_{h\to 0}\frac{f(x+h)-f(x)}{h} + \lim_{h\to 0}\frac{g(x+h)-g(x)}{h}\\
    &=f'(x) + g'(x)
\end{align*}

\begin{propositionbox}[pravilo o razlici funkcija]
    Derivacija razlike je jednaka razlici derivacija:

    $$
        (f(x)-g(x))' = f'(x) - g'(x)
    $$
\end{propositionbox}

Dokaz je identičan pravilu o derivaciji zbroja funkcija:

\begin{align*}
    &(f(x)-g(x))'=\\
    &=\lim_{h\to 0}\frac{f(x+h)-g(x+h)-(f(x)+g(x))}{h}\\
    &=\lim_{h\to 0}\frac{f(x+h)-f(x)-(g(x+h)-g(x))}{h}\\
    &=\lim_{h\to 0}\frac{f(x+h)-f(x)}{h} - \lim_{h\to 0}\frac{g(x+h)-g(x)}{h}\\
    &=f'(x) - g'(x)
\end{align*}

\newcolumn

\begin{propositionbox}[pravilo o deriviranju konstante]
    Derivacija konstante $c$ je jednaka nuli:

    $$
        \frac{d}{dx}(c) = 0
    $$
\end{propositionbox}

Zamjenimo konstantu $c$ sa funkcijom $f$ koja za svaki $x$ vraća vrijednost $c$, $f(x)=c$.
Iz toga slijedi:

\begin{align*}
    f'(x) =& \lim_{h\to 0}\frac{f(x+h)-f(x)}{h}\\
          =& \lim_{h\to 0}\frac{c - c}{h} = \lim_{h\to 0}0 = 0
\end{align*}

\begin{propositionbox}[pravilo o deriviranju umnoška funkcije s konstantom]
    Derivacija umnoška konstante i funkcije je jednaka umnošku te konstante i derivacije funkcije:
    $$
        (cf(x))' = cf'(x)
    $$
\end{propositionbox}

\begin{align*}
    (cf(x))'=&\lim_{h\to 0}\frac{cf(x+h)-cf(x)}{h}\\
            =&c\lim_{h\to 0}\frac{f(x+h)-f(x)}{h}\\
            =&cf'(x)
\end{align*}

\end{multicols}

\begin{propositionbox}[pravilo o deriviranju umnoška funkcija]
    $$
        (fg)'(x) = (f(x)\cdot g(x))' = f'(x)g(x) + f(x)g'(x)
    $$
\end{propositionbox}

\begin{align*}
    (fg)'(x)=&\lim_{h\to 0}\frac{cf(x+h)-cf(x)}{h}\\
    =&c\lim_{h\to 0}\frac{f(x+h)-f(x)}{h}\\
    =&cf'(x)
\end{align*}

\begin{propositionbox}[pravilo o deriviranju kvocijenta funkcija]
    $$
        \left(\frac{f(x)}{g(x)}\right)' = \frac{f'(x)g(x) - f(x)g'(x)}{(g(x))^2}
    $$
\end{propositionbox}

\begin{propositionbox}[pravilo o deriviranju inverzne funkcije]
    $$
        (f^{-1})'(x) = \frac{1}{f'(f^{-1}(x))}
    $$
\end{propositionbox}

\begin{propositionbox}[pravilo o deriviranju kompozicije funkcija]
    $$
        (f\circ g)'(x) = (f(g(x)))' = f'(g(x)) \cdot g'(x)
    $$
\end{propositionbox}

\begin{propositionbox}[pravilo o deriviranju eksponenta prirodnog broja]
    $$
        (e^x)' = e^x \cdot x'
    $$
\end{propositionbox}

\begin{propositionbox}[pravilo o deriviranju funkcije s potencijom]
    $$
        (x^{n})' = nx^{n-1}
    $$
\end{propositionbox}

\begin{propositionbox}[pravilo o deriviranju logaritamske funkcije]
    $$
        (\ln(u))' = \frac{u'}{u}
    $$
\end{propositionbox}

\subsection{Pravila deriviranja trigonometrijskih funkcija}

\begin{propositionbox}[pravilo o deriviranju sinusa]
    $$
        (\sin(x))' = \cos(x)
    $$
\end{propositionbox}

\begin{propositionbox}[pravilo o deriviranju kosinusa]
    $$
        (\cos(x))' = -\sin(x)
    $$
\end{propositionbox}

\begin{propositionbox}[pravilo o deriviranju tangensa]
    $$
        (\tan(x))' = \sec^2(x)
    $$
\end{propositionbox}

\begin{propositionbox}[pravilo o deriviranju kotangense]
    $$
        (\cot(x))' = -\csc^2(x)
    $$
\end{propositionbox}

\begin{propositionbox}[pravilo o deriviranju sekansa]
    $$
        (\sec(x))' = \sec(x) \tan(x)
    $$
\end{propositionbox}

\begin{propositionbox}[pravilo o deriviranju kosekansa]
    $$
        (\csc(x))' = -\csc(x) \cot(x)
    $$
\end{propositionbox}
