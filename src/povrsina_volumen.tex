\section{Površina i volumen}

\subsection{Površina ravninskog lika}

Površina ravninskog lika omeđenog krivuljama $y=f(x)$ i $y=g(x)$ te pravcima
$x=a$ i $x=b$, kao na slici, jednaka je
$$
    P=\int_a^b [f(x) - g(x)]\,dx = [F(x) - G(x)]\,\Big|_a^b.
$$

\begin{example}
    Izračunajte površinu lika omeđenog parabolom $y=x^2-3$ i pravcem $y=-x-1$.
\end{example}

\begin{example}
    Izračunajte površinu lika omeđenog:
    \begin{itemize}
        \item parabolama $y=4-x^2$ i $y=x^2-2x$,
        \item parabolom $y=2x-x^2$ i pravcem $y=-x$,
    \end{itemize}
\end{example}

% Za pravac je moguće primjeniti teorem srednje vrijednosti? Ima li to smisla?

\begin{example}
    Izračunajte površinu područja omeđenog krivuljama $y^2=x$ i $y=x-2$.
\end{example}

\subsection{Volumen rotacijskog tijela}

Graf funkcije $f$ rotiramo oko $x$-osi. Volumen tijela dobivenog rotacijom lika
određenog dijelom grafa funkcije $f$ te pravcima $x=a$, $x=b$ i $x$-osi, kao na
slici, jednak je
$$
    V=\pi \int_a^b f^2(x)\,dx.
$$

\begin{example}
    Izračunajte volumen tijela nastalog rotacijom oko $x$-osi lika omeđenog
    parabolom $y^2=4x$ i pravcem $x=1$.
\end{example}

\begin{example}
    Izračunajte volumen tijela nastalog rotacijom oko $x$-osi lika omedenog
    parabolom $y^2 = x$ i pravcem $x=4$.
\end{example}

\begin{example}
    Nadite volumen tijela nastalog rotacijom kružnice $x^2+y^2=r^2$ oko $x$-osi.
\end{example}

\begin{example}
    Nadite volumen tijela nastalog rotacijom kružnice $(x-2)^2+y^2=1$ oko
    $x$-osi.
\end{example}

\begin{example}
    Nadite volumen tijela nastalog rotacijom kružnice $(x-4)^2+y^2=4$ oko
    $x$-osi.
\end{example}

\begin{example}
    Nadite volumen tijela nastalog rotacijom elipse
    $\frac{x^2}{a^2}+\frac{y^2}{b^2}=1$ oko $x$-osi.
\end{example}
