\section{Krajnje vrijednosti}

\subsection{Minimum i maksimum}

Maksimum i minimum funkcije su, sukladno, najveća i najmanja vrijednost koju
funkcija proprima.

Maksimum i minimum generalno nazivamo \textbf{ekstremima} funkcije.

Mogu biti definirani za određeni interval (lokalni/relativni ekstremi) ili za
cijelu domenu (globalni/apsolutni ekstremi) funckije.

\begin{multicols}{2}
    
\begin{definition}[globalni maksimum]
    Funkcija realnih vrijednosti $f$ definirana na domeni $X$ ima
    \textbf{globalnu} (ili \textbf{apsolutnu}) \textbf{točku maksimuma} u $x^*$,
    ako je $f(x^*) \geq f(x)$ za svaki $x \in X$.

    \smallskip
    Vrijednost funkcije u toj točki se zove \textbf{maksimalna vrijednost}, a
    označava se sa $\max(f(x))$.
\end{definition}

\begin{definition}[globalni minimum]
    Funkcija realnih vrijednosti $f$ definirana na domeni $X$ ima
    \textbf{točku globalnog} (ili \textbf{apsolutnog}) \textbf{minimuma} u $x^*$,
    ako je $f(x^*) \leq f(x)$ za svaki $x \in X$.

    \smallskip
    Vrijednost funkcije u toj točki se zove \textbf{minimalna vrijednost}, a
    označava se sa $\min(f(x))$.
\end{definition}

\begin{definition}[lokalni maksimum]
    Ako je domena $X$ metrični prostor, onda kažemo da $f$ ima \textbf{točku
    lokalnog} (ili \textbf{relativnog}) \textbf{maksimuma} u $x^*$, ako postoji
    neki $\varepsilon > 0$ takav da je $f(x^*) \geq f(x)$ za svaki $x \in X$
    udaljen za $\varepsilon$ od $x^*$.
\end{definition}

\begin{definition}[lokalni minimum]
    Ako je domena $X$ metrični prostor, onda kažemo da $f$ ima \textbf{točku
    lokalnog} (ili \textbf{relativnog}) \textbf{minimuma} u $x^*$, ako postoji
    neki $\varepsilon > 0$ takav da je $f(x^*) \leq f(x)$ za svaki $x \in X$
    udaljen za $\varepsilon$ od $x^*$.
\end{definition}

\end{multicols}

I globalni i lokalni ekstremi mogu biti \textbf{strogi ekstremi}, ako su
jedinstveni unutar domene funkcije ili neke udaljenosti $\varepsilon$ od $x^*$ (sukladno).

\subsection{Infimum i supremum}

\textbf{Infimum} (skraćeno \textbf{inf}) podskupa $S$ djelomično složenog
skupa $P$ je najveći element u $P$ koji je manji ili jednak svakom elementu
$S$, ako takav element postoji.

\smallskip
\noindent
\textbf{Supremum} (skraćeno \textbf{sup}) podskupa $S$ djelomično složenog
skupa $P$ je najmanji element u $P$ koji je veći ili jednak svakom elementu
$S$, ako takav element postoji.

\begin{definition}[infimum]
    Donja međa podskupa $S$ parcijalno uređenog skupa $(P, \leq)$ je element $a$
    iz $P$ takav da vrijedi:
    $$
        a \leq x,\, \forall x \in S\,.
    $$

    Donja međa $a$ skupa $S$ se zove \textbf{infimum} (ili najveća donja međa,
    ili sastanak (engl. \textit{meet})) skupa $S$ ako je vrijednost $a$ veća od
    svih donjih međa $y$ skupa $S$ skoje su sadržane u $P$.
\end{definition}


\begin{definition}[supremum]
    Gornja međa podskupa $S$ parcijalno uređenog skupa $(P, \leq)$ je element
    $b$ iz $P$ za kojeg vrijedi:
    $$
        b \geq x,\, \forall x \in S\,.
    $$

    Gornja međa $b$ skupa $S$ se zove \textbf{supremum} (ili najmanja gornja
    međa, ili spoj (engl. \textit{join})) skupa $S$ ako je vrijednost $b$ manja
    od svih gornjih međa $z$ skupa $S$ skoje su sadržane u $P$.
\end{definition}

