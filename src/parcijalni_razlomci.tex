
\section{Parcijalni razlomci}

Pravu racionalnu funkciju možemo rastaviti na \textbf{parcijalne razlomke} ako
nam je poznata faktorizacija polinoma $Q_m$ u nazivniku.

Prilikom rastave na parcijalne razlomke vrijedi
\begin{itemize}
    \item svakom linearnom faktoru $(x-x_i)^{x_i}$ polinoma iz nazivnika odgovra
    izraz
    $$
      \frac{A_{i1}}{x-x_i} + \frac{A_{i2}}{(x-x_i)^2} + \dots + \frac{A_{in_i}}{(x-x_i)^{n_i}},
    $$
    \item svakom kvadratnom faktoru $(x^2+p_jx+q_j)^{m_j}$ polinoma iz nazivnika
    odgovara izraz
    $$
      \frac{B_{j1}x+C_{j1}}{x^2+p_jx+q_j} + \frac{B_{j2}x+C_{j2}}{(x^2+p_jx+q_j)^2} + \dots + \frac{B_{jm_j}x+C_{jm_j}}{(x^2+p_jx+q_j)^{m_j}}.
    $$
\end{itemize}

Nepoznate koeficijenate $A_{il}$, $B_{jk}$ i $C_{jk}$ nalazimo iz same
jednakosti rastava (množenjem s nazivnikom i izjednačavanjem koeficijenata uz
iste postencije od $x$).

\begin{example}
    Rastavite na parcijalne razlomke funkcije:
    \begin{enumerate}
        \item $R(x)=\frac{x+2}{x^2-1}$,
        \item $R(x)=\frac{2x^2+x+2}{x^3+x}$,
        \item $R(x)=\frac{1}{x^3+2x^2+x}$,
        \item $R(x)=\frac{x+1}{x^3-x^2}$.
    \end{enumerate}
\end{example}

\subsection{Integriranje parcijalnih razlomaka}

Integrale parcijalnih razlomaka svodimo na tablične integrale na sljedeći način:

\begin{enumerate}
    \item integral $\int \frac{dx}{(x-a)^n}$ supstitucijom $t=x-a$,
    \item integral $\int \frac{2x+p}{(x^2+px+q)^n}$ supstitucijom $t=x^2+px+q$.
\end{enumerate}

\begin{example}
    Izračunajte integral $\int R(x) dx$:
    \begin{enumerate}
        \item za racionalne funkcije $R(x)$ iz prethodnih primjera,
        \item za $R(x) = \frac{x^5+x^4-8}{x^3-4x}$
    \end{enumerate}
\end{example}
