\section{L'Hospitalovo pravilo}

L'Hospitalovo pravilo služi za izračunavanje limesa neodređenih oblika $\frac{0}{0}$
i $\frac{\infty}{\infty}$.

\begin{definitionbox}[L'Hospitalovo pravilo]
    Neka su zadane funkcije $f$ i $g$ neprekidne na $[a,b]$ za koje vrijedi:
    \begin{gather*}
        \lim_{x\to c}f(x) = \lim_{x\to c}g(x) = 0\\
        \text{ili}\\
        \lim_{x\to c}f(x) = \pm\infty\quad\text{i}\quad\lim_{x\to c}g(x)=\pm\infty
    \end{gather*}

    Pretpostavimo da postoje $f'$ i $g'$ na $\langle a,b\rangle$ osim možda u točki $c\in\langle a,b\rangle$,
    te da je $g'(x)\neq 0$ u okolini točke $c$. Ako postoji $\lim_{x\to c}\frac{f'(x)}{g'(x)}$ tada vrijedi:

    \center
    \begin{equation}
        \label{eq:lhospital}
        \tag{l'hospitalovo pravilo}
        \phantom{\text{(l'hospitalovo pravilo)}} % dunno why here and now...
        \lim_{x\to c}\frac{f(x)}{g(x)} = \lim_{x\to c}\frac{f'(x)}{g'(x)}
    \end{equation}
\end{definitionbox}

Pravilo je moguče primjeniti uzastopno ako su zadovoljeni uvjeti teorema za
derivacije funkcija $f$ i $g$.

\subsection{Neodređeni oblici}

Kod računanja limesa mogu se pojaviti i neodređeni oblici $0\cdot\infty$, $\infty - \infty$,
$0^0$, $1^\infty$ i $\infty^0$ koji se primjenom odgovarajućih transformacija svode na
jedan od dva oblika koji se mogu riješiti uz pomoć l'hospitalovog pravila.

\begin{center}
\begin{tblr}{
    colspec = {X[0.6,c,m]X[1.6,c,m]X[3,c,m]X[3,c,m]},
    stretch = 0,
    rowsep = 6pt,
    hlines = {colordefinitionfg, 0.5pt},
    vlines = {colordefinitionfg, 0.5pt},
    }
    \SetRow{bg=colordefinitionbg,font=\sffamily\bfseries}
    Oblik&Uvjeti&Transformacija u $\displaystyle\frac{0}{0}$
    &Transformacija u $\displaystyle\frac{\infty}{\infty}$\\
    $\frac{0}{0}$
    &{$\lim_{x \to c} f(x) = 0$,\\$\lim_{x \to c} g(x) = 0$}
    &-
    &$\displaystyle\lim_{x \to c} \frac{f(x)}{g(x)} = \lim_{x \to c} \frac{\frac{1}{g(x)}}{\frac{1}{f(x)}}$\\
    $\frac{\infty}{\infty}$
    &{$\lim_{x \to c} f(x) = \infty$,\\$\lim_{x \to c} g(x) = \infty$}
    &$\displaystyle\lim_{x \to c} \frac{f(x)}{g(x)} = \lim_{x \to c} \frac{\frac{1}{g(x)}}{\frac{1}{f(x)}}$
    &-\\
    $0\cdot\infty$
    &{$\lim_{x \to c} f(x) = 0$,\\$\lim_{x \to c} g(x) = \infty$}
    &$\displaystyle\lim_{x \to c} f(x)g(x) = \lim_{x \to c} \frac{f(x)}{\frac{1}{g(x)}}$
    &$\displaystyle\lim_{x \to c} f(x)g(x) = \lim_{x \to c} \frac{g(x)}{\frac{1}{f(x)}}$\\
    $\infty - \infty$
    &{$\lim_{x \to c} f(x) = \infty$,\\$\lim_{x \to c} g(x) = \infty$}
    &$\displaystyle\lim_{x \to c} (f(x) - g(x)) = \lim_{x \to c} \frac{\frac{1}{g(x)} - \frac{1}{f(x)}}{\frac{1}{fg(x)}}$
    &$\displaystyle\lim_{x \to c} (f(x) - g(x)) = \ln \lim_{x \to c} \frac{e^{f(x)}}{e^{g(x)}}$\\
    $0^0$
    &{$\lim_{x \to c} f(x) = 0^+$,\\$\lim_{x \to c} g(x) = 0$}
    &$\displaystyle\lim_{x \to c} f(x)^{g(x)} = \exp \lim_{x \to c} \frac{g(x)}{\frac{1}{\ln f(x)}}$
    &$\displaystyle\lim_{x \to c} f(x)^{g(x)} = \exp \lim_{x \to c} \frac{\ln f(x)}{\frac{1}{g(x)}}$\\
    $1^\infty$
    &{$\lim_{x \to c} f(x) = 1$,\\$\lim_{x \to c} g(x) = \infty$}
    &$\displaystyle\lim_{x \to c} f(x)^{g(x)} = \exp \lim_{x \to c} \frac{\ln f(x)}{\frac{1}{g(x)}}$
    &$\displaystyle\lim_{x \to c} f(x)^{g(x)} = \exp \lim_{x \to c} \frac{g(x)}{\frac{1}{\ln f(x)}}$\\
    $\infty^0$
    &{$\lim_{x \to c} f(x) = \infty$,\\$\lim_{x \to c} g(x) = 0$}
    &$\displaystyle\lim_{x \to c} f(x)^{g(x)} = \exp \lim_{x \to c} \frac{g(x)}{\frac{1}{\ln f(x)}}$
    &$\displaystyle\lim_{x \to c} f(x)^{g(x)} = \exp \lim_{x \to c} \frac{\ln f(x)}{\frac{1}{g(x)}}$\\
\end{tblr}
\end{center}
