\section{Tok funkcije}

\subsection{Domena}

Funkcija $f(x)$ je definirana za sve realne brojeve $x\in\mathbb{R}$, osim ako
njena vrijednost nije definirana za neki $x$. Vrijednost funkcije za $f(x)$ je
definirana ako spada u očekivanu kodomenu ($\mathcal{C}$) funkcije.

To se provjerava ovisno o uvjetima izraza korištenih u funkciji, no učestali
uzroci su:
\begin{itemize}
    \item vrijednosti $x$ za koje nazivnik razlomka u funkciji poprima vrijednost $0$,
    \item pojavljivanje korijena negativnog broja (ako kodomena ne uključuje imaginarne brojeve),
    \item negativan argument logaritma ili argument jednak nuli,
    \item $\dots$
\end{itemize}

\noindent
Domena funkcije je skup svih realnih brojeva $\mathbb{R}$ osim vrijednosti $x$
koje uvrštene u funkciju $f$ daju nevaljale/neodređene rezultate.

\subsection{Nultočke}

Nultočke funkcije $f$ su vrijednosti $x$ za koje funkcija poprima vrijednost $0$.
U grafičkom prikazu to znači da prelazi na tim točkama preko x-osi.

Za određivanje nultočaka, rješavamo jednadžbu:
$$
f(x) = 0
$$

\subsection{Intervali monotonosti}

Interval monotonosti funkcije je interval na kojem funkcija kontinuirano raste
ili pada.

\begin{theorembox}
    Neka je funkcija $f$ derivabilna na $\langle a,b \rangle \subseteq \mathcal{D}_f$.
    Tada vrijedi:
    \begin{itemize}
        \item $f$ raste na $\langle a,b \rangle \Leftrightarrow f'(x) \geq 0, \forall x \in \langle a,b \rangle$,
        \item $f$ pada na $\langle a,b \rangle \Leftrightarrow f'(x) \leq 0, \forall x \in \langle a,b \rangle$.
    \end{itemize}
\end{theorembox}

Točke $c$ za koje vrijedi $f'(c)=0$ zovemo \textbf{stacionarnim točkama}
funkcije $f$.

Točku $c\in \mathcal{D}_f$ nazivamo \textbf{kritičnom točkom} funkcije $f$ ako
je $c$ stacionarna točka od $f$ ili ako $f$ nije derivabilna u točki $c$.

\subsection{Lokalni ekstremi}

\begin{definitionbox}
    Točka $(c, f(c))$ je \textbf{lokalni minimum} funkcije $f$ ako postoji
    interval $\langle a,b \rangle$ koji sadrži točku $c$ takav da
    $f(c)<f(x),\forall x \in \langle a,b \rangle, x\neq c$.
    Točka $(c, f(c))$ je \textbf{lokalni maksimum} funkcije $f$ ako postoji
    interval $\langle a,b \rangle$ koji sadrži točku $c$ takav da
    $f(c)>f(x),\forall x \in \langle a,b \rangle, x\neq c$.
\end{definitionbox}

U točki ekstrema $(c, f(c))$, tangenta na graf funkcije (ako postoji) je
paralelna x-osi, odnosno njen koeficijent smjera $f'(c) = 0$.

Funkcija $f$ ima lokalni ekstrem u kritičnoj točki $c$ ako i samo ako se
predznak funkcije $f'$ na području lijevo od $c$ razlikuje od predznaka funkcije
$f'$ na području desno od $c$.

\begin{theorembox}
    Neka funkcija $f$ u nekoj okolini točke $c$ ima neprekidne derivacije od
    reda $n, n\in\mathbb{N},n\geq 2$.
    Neka vrijedi
    $$
    f'(c)=f''(c)=\cdots=f^{(n-1)}(c)=0,\qquad f^{(n)}(c)\neq 0.
    $$

    \begin{itemize}
        \item Ako je $n$ paran broj, funkcija $f$ ima lokalni ekstrem u točki $c$, i to:
        \begin{itemize}
            \item lokalni minimum ako je $f^{(n)}(c)<0$, a
            \item lokalni maksimum ako je $f^{(n)}(c)>0$.
        \end{itemize}
        \item Ako je $n$ neparan broj, funkcija $f$ \textit{nema} lokalni
        ekstrem u $c$.
    \end{itemize}
\end{theorembox}

\subsection{Konveksnost i konkavnost}

Funkcija $f$ je \textbf{konveksna} na intervalu $\langle a,b \rangle\subseteq\mathcal{D}_f$
ako vrijedi:

$$
f\left(\frac{x_1+x_2}{2}\right) \leq \frac{f(x_1)+f(x_2)}{2},\qquad\forall x_1,x_2 \in \langle a,b \rangle,\qquad x_1\neq x_2
$$

Ako je navedena nejednakost stroga, funkcija je \textbf{strogo konveksna}.


Funkcija $f$ je \textbf{konkavna} na intervalu $\langle a,b \rangle\subseteq\mathcal{D}_f$
ako vrijedi:

$$
f\left(\frac{x_1+x_2}{2}\right) \geq \frac{f(x_1)+f(x_2)}{2},\qquad\forall x_1,x_2 \in \langle a,b \rangle,\qquad x_1\neq x_2
$$

Ako je navedena nejednakost stroga, funkcija je \textbf{strogo konkavna}.

\begin{theorembox}
    Neka funkcija $f$ ima drugu derivaciju na $\langle a,b \rangle\subseteq\mathcal{D}_f$.
    Tada vrijedi:
    \begin{itemize}
        \item $f$ je strogo konveksna na $\langle a,b \rangle \Leftrightarrow f''(x)>0,\forall x \in \langle a,b \rangle$,
        \item $f$ je strogo konkavna na $\langle a,b \rangle \Leftrightarrow f''(x)<0,\forall x \in \langle a,b \rangle$.
    \end{itemize}
\end{theorembox}

Točku $c\in\mathcal{D}_f$ u kojoj se mijenja predznak funkcije $f''$ nazivamo
\textbf{točkom infleksije} funkcije $f$. Za točku infleksije vrijedi $f''(c) = 0$.

\begin{theorembox}
    Neka funkcija $f$ u nekoj okolini točke $c$ ima neprekidne derivacije od
    reda $n$, gdje je $n\in\mathbb{N}, n\geq 3$.
    Neka vrijedi:
    $$
    f''(c)=f'''(c)=\cdots=f^{(n-1)}(c)=0,\qquad f^{(n)}(c) \neq 0.
    $$

    \begin{itemize}
        \item Ako je $n$ neparan broj, $c$ \textbf{je} točka infleksije funkcije $f$.
        \item Ako je $n$ paran broj, $c$ \textbf{nije} točka infleksije funkcije $f$.
    \end{itemize}
\end{theorembox}

\subsection{Asimptote}

Ako je
$$
\lim_{x\to c\pm}f(x)=\pm\infty,
$$
kažemo da je pravac $x=c$ \textbf{vertikalna asimptota} funkcije $f$.

Ako je
$$
\lim_{x\to\pm\infty}\frac{f(x)}{x}=k,\qquad \lim_{x\to\pm\infty}(f(x)-kx)=l,
$$
kažemo da je pravac $y=kx+l$ \textbf{kosa asimptota} funkcije $f$.

Za određivanje \textbf{horizontalne asimptote} uspoređujemo stupnjeve polinoma
funkcije koja je izražena kao razlomak:
$$
f(x)=\frac{ax^n+\dots}{bx^m+\dots}
$$
\begin{itemize}
    \item Ako je $n<m$ onda je horizontalna asimptota x-os ($y=0$).
    \item Ako je $n=m$ onda je horizontalna asimptota $\displaystyle y=\frac{a}{b}$.
    \item Ako je $n>m$ onda horizontalna asimptota ne postoji.
\end{itemize}

U slučaju gdje funkcija nije razlomak, pretpostavljamo da je nazivnik $1$, te
prema navedenim pravilima horizontalna asimptota ne postoji.

\subsection{Primjer određivanja toka funkcije}


