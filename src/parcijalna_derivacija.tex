\subsection{Parcijalna derivacija}

\textbf{Funkcija dvije varijable} je funkcija

$$
f: \mathrm{S} \to \mathbb{R}, \mathrm{S} \subseteq \mathbb{R}^2, (x, y) \mapsto f(x,y).
$$

\textbf{Prirodna domena} funkcije dvije varijable $f$ je najveći podskup od
$\mathbb{R}^2$ na kojem formula $f(x,y)$ ima smisla.

\textbf{Graf} funkcije dvije varijable $f$ je skup
$$
  \{(x,y,z)\in\mathbb{R}^3\ |\ z=f(x,y),\ (x,y)\in\mathrm{S}\}.
$$

\begin{example}
  Odredite i skicirajte prirodnu domenu funkcije $f(x,y)=\ln(x+y)$.
\end{example}

Parcijalna derivacija funkcije $f(x,y,\dots)$ po varijabli $x$ se označava kao:
$$
f_x, f_x', \partial_x f, \frac{\partial}{\partial x}f, \text{ili}\ \frac{\partial f}{\partial x},
$$
te opisuje promjenu funkcije u $x$-smjeru.

\begin{definition}
    \label{def:diff_partial}
    Neka je $\mathbf{S}$ otvoreni podskup skupa $\mathbb{R}^n$ te neka je
    funkcija $f:\mathbf{S}\to \mathbb{R}$.

    Parcijalna derivacija $f$ u točki $\mathbf{a}=(a_1, a_2, \dots,
    a_n)\in\mathbf{S}$ po $i$-toj varijabli $x_i$ je:
    
    \begin{align*}
        \frac{\partial }{\partial x_i }f(\mathbf{a}) & = \lim_{h \to 0} \frac{f(a_1, \ldots , a_i+h, \ldots ,a_n) -
        f(a_1, \ldots, a_i, \dots ,a_n)}{h} \\ 
        & = \lim_{h \to 0} \frac{f(\mathbf{a}+h \cdot \mathbf{e_i}) - f(\mathbf{a})}{h}\,,
    \end{align*}

    gdje $e_i$ predstavlja jedinični vektor u $i$-tom smjeru, a $h$ je pomak u
    $i$-tom smjeru.
\end{definition}

Neka je $f(x,y)$ funkcija dvije varijable. \textbf{Parcijalna derivacija
funkcije $f$ po varijabli $x$} je
$$
  \frac{\partial f}{\partial x} = \lim_{\Delta x\to 0} \frac{f(x+\Delta x,y)-f(x,y)}{\Delta x}.
$$

\textbf{Parcijalna derivacija funkcije $f$ po varijabli $y$} je
$$
  \frac{\partial f}{\partial y} = \lim_{\Delta y\to 0} \frac{f(x+\Delta y,y)-f(x,y)}{\Delta y}.
$$

\subsubsection{Parcijalna derivacija višeg reda}

Parcijalnu derivaciju funkcije višeg reda dobivamo analogno derivaciji višeg
reda.

Neka je $f(x,y)$ funkcija dvije varijable.

\begin{align*}
  \frac{\partial f}{\partial x^2} &= \frac{\partial}{\partial x} (\frac{\partial f}{\partial x}),\\
  \frac{\partial f}{\partial x \partial y} &= \frac{\partial}{\partial x} (\frac{\partial f}{\partial y}),\\
  \frac{\partial f}{\partial y \partial x} &= \frac{\partial}{\partial y} (\frac{\partial f}{\partial x}),\\
  \frac{\partial f}{\partial y^2} &= \frac{\partial}{\partial y} (\frac{\partial f}{\partial y}),\\
\end{align*}

\begin{example}
  Odredite parcijalne derivacije drugog reda funkcije $f(x,y)=x^3y+x\sin y$.
\end{example}

\subsection{Gradijent}

\textbf{Gradijent funkcije} $f(x,y,\dots)$ je vektor čije su komponente
parcijalne derivacije funkcije $f$ po svakoj varijabli.

Često se označava kao $\nabla f$, te je za funkciju sa dvije varijable $f(x,y)$ jednak:
$$
  \nabla f(x,y) = \left[\frac{\partial}{\partial x}\right]\hat{i} + \left[\frac{\partial}{\partial y}\right]\hat{j}.
$$

No vrijedi i analog za funkcije s večim brojem varijabli, te se generalno, za
funkciju $f(x_1, \dots, x_n)$, sa jediničnim vektorima
$\hat{e}_1,\dots,\hat{e}_n$ može zapisati:
$$
  \nabla f(x_1, \dots, x_n) = \sum_{i=1}^n \left[\frac{\partial}{\partial x_i}\right]\hat{e}_i = \left[\frac{\partial}{\partial x_1}\right]\hat{e}_1 + \left[\frac{\partial}{\partial x_2}\right]\hat{e}_2 + \dots + \left[\frac{\partial}{\partial x_n}\right]\hat{e}_n.
$$

\textbf{Totalni diferencijal prvog reda} funkcije $f$ je definiran s
$$
df(x,y) = \frac{\partial f}{\partial x}dx + \frac{\partial f}{\partial y}dy.
$$

\subsection{Tangencijalna ravnina}

Tangencijalna ravnina funkcije $f(x,y)$ u točki $(x_0, y_0)$ je ravnina koja
sadrži tangente na funkciju $f$ u točki $(x_0, y_0)$.

\tikzset
{ surface/.style={draw=blue,shading=ball,ball color=cyan!60!blue,fill
  opacity=0.8}, plane/.style={draw=orange,fill=orange,fill opacity=0.6},
  vector/.style={draw=magenta,thick,-latex} }

% Source: https://tex.stackexchange.com/questions/679299/i-have-to-reproduce-a-figure-of-tangent-plane-to-a-surface-in-mathbbr3
\begin{center}
\begin{tikzpicture}[line cap=round,line join=round,3d view={110}{20},scale=1.5]
  % dimensions
  \def\r{2}    % sphere radius
  \def\c{0.5}  % center C position
  \def\cx{0.5} % curve x position
  \pgfmathsetmacro\ra{sqrt(\r*\r-\c*\c)} % radius, arcs in the sphere
  
  \pgfmathsetmacro\rc{sqrt(\r*\r-(\c+\cx)*(\c+\cx))} % radius, curve
  \pgfmathsetmacro\axy{asin(\c/\r)}                  % angles in xy plane
  \pgfmathsetmacro\ayz{asin(\c/\ra)}                 % angles in yz plane 
  \pgfmathsetmacro\ar{asin((\c+\cx)/\r} % rotation angle for plane and vectors
  % coordinates
  \coordinate (C) at (-\c,\c,0); % sphere center
  \coordinate (O) at (0,0,0); \coordinate (P) at
  (\cx,\c,{sqrt(\r*\r-(\c+\cx)*(\c+\cx))}); \coordinate (G) at ($(C)!1.7!(P)$);
  % gradient
  % axes
  \draw[-latex] (O) -- (\r+1,0,0) node[left]  {$x$}; \draw[-latex] (O) --
  (0,\r+1,0) node[right] {$y$}; \draw[-latex] (O) -- (0,0,\r+1) node[above]
  {$z$};
  % surface
  \draw[surface] (C) ++ (-\axy:\r) arc (-\axy:90-\axy:\r) {[canvas is yz plane
      at x=0] arc (0:90+\ayz:\ra)} {[canvas is xz plane at y=0] arc
      (90-\ayz:0:\ra)};
  % curve
  \draw[canvas is yz plane at x=\cx,yellow] (\c+\rc,0) arc
  (0:{90+asin(\c/\rc)}:\rc);
  % plane
  \draw[plane,shift={(P)},rotate around y=\ar,canvas is xy plane at z=0]
  (-1,-1.5) -|++ (2,3) -| cycle;
  % Point P
  \fill (P) circle (0.4mm) node[below] {$P$};
  % vectors
  \begin{scope}[shift={(P)},rotate around y=270+\ar,canvas is xy plane at z=0]
    \draw[vector]  (0.04,0) -- (G) node[above right,black] {$\nabla
    F(x_0,y_0,z_0)$};
  \end{scope}
\end{tikzpicture}
\end{center}

$$
  P(x_0, y_0, z_0) \Rightarrow \frac{\partial f}{\partial x}(x_0, y_0)(x-x_0) + \frac{\partial f}{\partial y}(x_0,y_0)(y-y_0) - (z-z_0) = 0
$$

Normala tangencijalne ravnine je gradijent kao što je prikazano na slici.

\begin{example}
  Odredite jednadžbu tangencijalne ravnine na plohu $z=x^2+y^2$ u točki $(1, -2,
  z_0)$.
\end{example}