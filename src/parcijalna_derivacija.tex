\section{Parcijalna derivacija}

\textbf{Funkcija dvije varijable} je funkcija

$$
f: \mathrm{S} \to \mathbb{R}, \mathrm{S} \subseteq \mathbb{R}^2, (x, y) \mapsto f(x,y).
$$

\textbf{Prirodna domena} funkcije dvije varijable $f$ je najveći podskup od
$\mathbb{R}^2$ na kojem formula $f(x,y)$ ima smisla.

\textbf{Graf} funkcije dvije varijable $f$ je skup
$$
  \{(x,y,z)\in\mathbb{R}^3\ |\ z=f(x,y),\ (x,y)\in\mathrm{S}\}.
$$

\begin{example}
  Odredite i skicirajte prirodnu domenu funkcije $f(x,y)=\ln(x+y)$.
\end{example}

Parcijalna derivacija funkcije $f(x,y,\dots)$ po varijabli $x$ se označava kao:
$$
f_x, f_x', \partial_x f, \frac{\partial}{\partial x}f, \text{ili}\ \frac{\partial f}{\partial x},
$$
te opisuje promjenu funkcije u $x$-smjeru.

\begin{definition}
    \label{def:diff_partial}
    Neka je $\mathbf{S}$ otvoreni podskup skupa $\mathbb{R}^n$ te neka je
    funkcija $f:\mathbf{S}\to \mathbb{R}$.

    Parcijalna derivacija $f$ u točki $\mathbf{a}=(a_1, a_2, \dots,
    a_n)\in\mathbf{S}$ po $i$-toj varijabli $x_i$ je:
    
    \begin{align*}
        \frac{\partial }{\partial x_i }f(\mathbf{a}) & = \lim_{h \to 0} \frac{f(a_1, \ldots , a_i+h, \ldots ,a_n) -
        f(a_1, \ldots, a_i, \dots ,a_n)}{h} \\ 
        & = \lim_{h \to 0} \frac{f(\mathbf{a}+h \cdot \mathbf{e_i}) - f(\mathbf{a})}{h}\,,
    \end{align*}

    gdje $e_i$ predstavlja jedinični vektor u $i$-tom smjeru, a $h$ je pomak u
    $i$-tom smjeru.
\end{definition}

Neka je $f(x,y)$ funkcija dvije varijable. \textbf{Parcijalna derivacija
funkcije $f$ po varijabli $x$} je
$$
  \frac{\partial f}{\partial x} = \lim_{\Delta x\to 0} \frac{f(x+\Delta x,y)-f(x,y)}{\Delta x}.
$$

\textbf{Parcijalna derivacija funkcije $f$ po varijabli $y$} je
$$
  \frac{\partial f}{\partial y} = \lim_{\Delta y\to 0} \frac{f(x+\Delta y,y)-f(x,y)}{\Delta y}.
$$

\subsection{Parcijalna derivacija višeg reda}

Parcijalnu derivaciju funkcije višeg reda dobivamo analogno derivaciji višeg
reda.

Neka je $f(x,y)$ funkcija dvije varijable.

\begin{align*}
  \frac{\partial f}{\partial x^2} &= \frac{\partial}{\partial x} (\frac{\partial f}{\partial x}),\\
  \frac{\partial f}{\partial x \partial y} &= \frac{\partial}{\partial x} (\frac{\partial f}{\partial y}),\\
  \frac{\partial f}{\partial y \partial x} &= \frac{\partial}{\partial y} (\frac{\partial f}{\partial x}),\\
  \frac{\partial f}{\partial y^2} &= \frac{\partial}{\partial y} (\frac{\partial f}{\partial y}),\\
\end{align*}

\begin{example}
  Odredite parcijalne derivacije drugog reda funkcije $f(x,y)=x^3y+x\sin y$.
\end{example}
