\section{Integral}

\subsection{Primitivna funkcija}
Neka je $f: \langle a, b \rangle \to \mathbb{R}$ funkcija.
\textbf{Primitivna funkcija} funkcije $f$ na $\langle a, b \rangle$ je svaka
funkcija $F: \langle a, b \rangle \to \mathbb{R}$ sa svojstvom:

$$
F'(x) = f(x),\quad\forall x \in \langle a, b \rangle.
$$

\begin{theorem}
    Neka je $f: \langle a, b \rangle \to \mathbb{R}$ funkcija, i $F$ i $G$
    bilo koje dvije primitivne funkcije od $f$ na $\langle a, b \rangle$. Tada
    postoji konstanta $C\in\mathbb{R}$ takva da vrijedi:

    $$
    F(x) = G(x) + C,\quad\forall x \in \langle a, b \rangle.
    $$
\end{theorem}

\begin{theorem}[Lagrangeov teorem srednje vrijednosti]
    Neka je funkcija $f$ neprekidna na segmentu $[a, b]$ i derivabilna na
    $\langle a, b\rangle$. Tada postoji barem jedna točka $c \in\langle a,
    b\rangle$ u kojoj je $f'(c)=\frac{f(b)-f(a)}{b-a}$.
\end{theorem}

\begin{example}
    Odredite jednu primitivnu funkciju sljedećih funkcija:
    \begin{enumerate}
        \item $f(x) = \sin x$,
        \item $f(x) = e^x$,
        \item $f(x) = {\frac{1}{x}}$,
        \item $f(x) = {\frac{1}{\sqrt{x}}}$.
    \end{enumerate}
\end{example}

\begin{example}
    Odredite primitivnu funkciju $F(x)$ funkcije:
    $$
        f(x) = \begin{cases}
            2, & x < 0, \\
            e^x, & x \geq 0.
        \end{cases}
    $$

    koja je neprekidna i za koju vrijedi $F(-1) = 4$.
\end{example}

\subsection{Neodređeni integral}

\textbf{Neodređeni integral} funkcije $f$, s oznakom $\int f(x)dx$, je skup svih
primitivnih funkcija funkcije $f(x)$. Vrijedi:

$$
\int f(x)dx = \{F(x) + C\ |\ C\in\mathbb{R}\},
$$

pri čemu $F(x)$ označava primitivnu funkciju funkcije $f(x)$, a $C$ konstantu.

Prema tome su neodređeni integral i derivacija inverzne operacije te su povezane
sljedećim relacijama:

\begin{align*}
    \left(\int f(x) dx\right)' &= f(x),\\
    \int F'(x) dx &= F(x) + C.
\end{align*}

Derivacija integrala jednaka je podintegralnoj funkciji, odnosno integriranje
poništava deriviranje do na konstantu.

\subsection{Direktna integracija}

Budući da je neodređeni integral "\textit{antiderivacija}", čitanjem tablice
derivacija u obrnutom smjeru dobivamo tablicu integrala.

Iz svojstva linearnosti derivacije slijedi \textbf{svojstvo linearnosti}
neodređenog integrala:

$$
\int \left[af(x) + bg(x)\right] dx = a \int f(x) dx + b \int g(x) dx.
$$

\textbf{Direktna integracija} svodi se na primjenu tablice integrala i svojstva
linearnosti.

\begin{example}
    Izračunajte integral:

    \begin{enumerate}
        \item $\displaystyle \int \left(4 cos(x) + \frac{1}{2} x^3 - 3\right)$
    \end{enumerate}
\end{example}

\subsection{Metoda supstitucije}

\textbf{Metoda supstitucije}, također nazvana i \textbf{u-supstitucija} je
postupak u kojem se zamjenom stare varijable integracije novom, neodređeni
integral transformira u novi oblik. Primjenjuje se kada želimo integrirati neku
kompozicuju funkcija.

Izvodi se iz pravila za derivaciju kompozicije funkcija.

\bigskip
\noindent
Za primjenu metode supstitucije potreban je integral oblika:
$$
\int f[g(x)]g'(x)dx,
$$

\noindent
a primjenjujemo ju tako da:
\begin{enumerate}
    \item ako je potrebno, uređujemo funkciju kako bi mogli primjeniti metodu
    supstitucije,
    \item odabiremo novu varijablu integracije $u = g(x)$, pri čemu je $g(x)$
    derivabilna funkcija,
    \item određujemo $du = g'(x)dx$,
    \item zamjenom varijabli u integralu dobivamo novi integral u kojem je
    integrirana varijabla $u$, a koji je jednostavnijeg oblika: $\displaystyle
    \int f[g(x)]g'(x)dx = \int f(u)du,$
    \item rješavanjem tog integrala dobivamo: $\displaystyle \int f(u)du = F(u)
    + C$, te konačno,
    \item vraćamo se na staru varijablu integracije $x$: $F(u) + C = F[g(x)] +
    C.$
\end{enumerate}

\begin{example}
    Izračunajte integrale:
    \begin{enumerate}
        \item $\displaystyle \int \cos (5x) dx$,
        \item $\displaystyle \int \sqrt{2x + 3} dx$,
        \item $\displaystyle \int \frac{\ln x}{x} dx$,
        \item $\displaystyle \int (3x^2+2x)e^{x^3+x^2} dx$,
    \end{enumerate}
\end{example}

\begin{align*}
\int \cos 5x\,dx &= \begin{Bmatrix} u = 5x \\ du = (5x)'dx = 5dx \\ dx = \frac{du}{5} \end{Bmatrix} = \int \cos u\,du \\
                 &= \frac{1}{5}\sin u + C = \frac{1}{5}\sin 5x + C, \qquad C \in \mathbb{R}
\end{align*}
