\subsection{Određeni integral}

\textbf{Određeni integral} je broj dobiven izračunavanjem integrala na nekom
intervalu.

\begin{definition}
    Ako postoji granična vrijednost zbroja $\sum_{i=1}^n f(c_i)\Delta x_i$ kada $n \to \infty$, $\Delta x_i \to 0$ dobivenu vrijednost zovemo \textbf{određeni integral}:

    $$
        \int_a^b f(x)dx = \lim_{\substack{n \to \infty \\ \Delta x_i \to 0}} \sum_{i=1}^n f(c_i)\Delta x_i
    $$
\end{definition}

Površina lika omeđenog grafom funkcije $f$, segmentom $[a, b]$ na $x$-osi te pravcima $x=a$ i $x=b$ je jednaka
$$
    P=\int_a^b |f(x)| dx.
$$

Površina između dvije funkcije $f$ i $g$ na intervalu $[a, b]$ je jednaka razlici površina segmenata $P$.

\subsubsection{Svojstva određenog integrala}

\begin{enumerate}
    \item $\displaystyle \int_a^a f(x)dx = 0$,
    \item $\displaystyle \int_b^a f(x)dx = -\int_a^b f(x)dx$,
    \item $\displaystyle \int_a^b f(x)dx = \int_a^c f(x)dx + \int_c^b f(x)dx$,
    \item linearnost: $\displaystyle \int_a^b (\alpha f(x) + \beta g(x))dx = \alpha \int_a^b f(x)dx \pm \beta \int_a^b g(x)dx$,
    \item monotonost: $f(x) \leq g(x) \implies \displaystyle \int_a^b f(x)dx \leq \int_a^b g(x)dx$ ako je $f(x) \leq g(x)$,
    \item nejednakost trokuta: $\displaystyle \left|\int_a^b f(x)dx\right| \leq \int_a^b |f(x)|dx$,
    \item teorem srednje vrijednosti
\end{enumerate}

\begin{theorem}[teorem srednje vrijednosti]
    Ako je funkcija $f(x)$ neprekidna na intervalu $[a, b]$, tada postoji točka $c \in \langle a, b\rangle$ takva da je
    $$
        f(c) = \frac{1}{b-a}\int_a^b f(x)dx.
    $$
\end{theorem}

\subsection{Newton-Leibnizova formula}

Neka je funkcija $f: [a,b] \mapsto \mathbb{R}$ integrabilna na $[a,b]$ i neka za nju postoji primitivna funkcija $F: [a,b] \mapsto \mathbb{R}$ takva da je $F'(x)=f(x)$ za svaki $x\in \langle a, b \rangle$. Tada vrijedi \textbf{Newton-Leibnizova formula}:

$$
  \int_a^b f(x)dx=F(b)-F(a).
$$

Newton-Leibnizova formula povezuje određeni integral s neodređenim te se često zapisuje u obliku
$$
  \int_a^b f(x)dx=F(x)\Big|_a^b.
$$

\begin{example}
    Na segmentu $[1,4]$ zadana je funkcija $f(x)=x^2$.
    Izračunajte površinu koju graf funkcije $f$ zatvara s $x$-osi nad promatranim segmentom.
\end{example}


\begin{example}
    Izračunajte:

    \begin{enumerate}
        \item $\displaystyle \int_e^{e^2} \frac{\ln x}{x} dx$,
        \item $\displaystyle \int_1^{e} 2x \ln x dx$.
    \end{enumerate}
\end{example}
