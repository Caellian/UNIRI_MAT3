\subsection{Lokalni ekstremi funckija više varijabli}

\begin{definition}[lokalni maksimum funkcije dviju varijabli]
    Za funkciju $f$ dviju varijabli kažemo da ima lokalni maksimum u točki
    $(x_0, y_0)$ ako postoji okolina te točke takva da za sve $(x,y)$ iz te
    okoline vrijedi $f(x_0,y_0) \geq f(x,y)$.
\end{definition}

\begin{definition}[lokalni minimum funkcije dviju varijabli]
    Za funkciju $f$ dviju varijabli kažemo da ima lokalni minimum u točki $(x_0,
    y_0)$ ako postoji okolina te točke takva da za sve $(x,y)$ iz te okoline
    vrijedi $f(x_0,y_0) \leq f(x,y)$.
\end{definition}

Ako funckija $f$ ima lokalni ekstrem u točki $(x_0, y_0)$ i ako u toj točki
postoje parcijalne derivacije prvog reda, onda mora vrijediti
$$
\frac{\partial f}{\partial x} (x_0, y_0) = \frac{\partial f}{\partial y} (x_0, y_0) = 0\,.
$$

Točke koje zadovoljavaju zaj uvjet zovemo \textbf{stacionarnim točkama}.

Stacionarna točka može biti:
\begin{itemize}
    \item točka lokalnog maksimuma,
    \item točka lokalnog minimuma, ili
    \item sedlasta točka.
\end{itemize}

\begin{definition}[sedlasta točka]
    Sedlasta točka je točka na površini grafa funkcije $f$ za koju su nagibi
    (derivacije) u okomitim smjerovima jednaki nuli, ali koja nije lokalni
    ekstrem funkcije.
\end{definition}

\subsubsection{Kriterij lokalnog ekstrema}

\begin{gather*}
    A = \frac{\partial^2 f}{\partial^2 x}\,,\quad
    B = \frac{\partial^2 f}{\partial y \partial x} = \frac{\partial^2 f}{\partial x \partial y}\,,\quad
    C = \frac{\partial^2 f}{\partial^2 y}\,\\
    \Delta = AC - B^2\,.
\end{gather*}

\noindent
Ako je
\begin{enumerate}
    \item $\Delta > 0$, onda je $(x_0,y_0)$ točka lokalnog ekstrema, i to
    \begin{itemize}
        \item maksimuma ako je $A < 0$,
        \item minimuma ako je $A > 0$.
    \end{itemize}
    \item $\Delta < 0$, onda je $(x_0, y_0)$ sedlasta točka.
    \item $\Delta = 0$, onda kriterij ne daje odluku.
\end{enumerate}

\begin{example}
    Provjeri ima li funckija $f(x,y) = 3x^2y+y^3-3x^2-3y^2+2$ sedlaste točke.
\end{example}

