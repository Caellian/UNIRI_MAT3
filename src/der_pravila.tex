\subsection{Pravila deriviranja}

\begin{multicols}{2}

\begin{propositionbox}[pravilo o zbroju funkcija]
    Derivacija zbroja je jednaka zbroju derivacija:

    $$
        (f(x)+g(x))' = f'(x) + g'(x)
    $$
\end{propositionbox}

Raspisujemo zbroj funkcija kroz \fancyeq[definiciju za][derivaciju funkcije]{def:diff},
uređujemo nazivnik i primjenjujemo pravilo o zbroju limesa:

\begin{align*}
    &(f(x)+g(x))'=\\
    &=\lim_{h\to 0}\frac{f(x+h)+g(x+h)-(f(x)+g(x))}{h}\\
    &=\lim_{h\to 0}\frac{f(x+h)-f(x)+g(x+h)-g(x)}{h}\\
    &=\lim_{h\to 0}\frac{f(x+h)-f(x)}{h} + \lim_{h\to 0}\frac{g(x+h)-g(x)}{h}\\
    &=f'(x) + g'(x)
\end{align*}

\begin{propositionbox}[pravilo o razlici funkcija]
    Derivacija razlike je jednaka razlici derivacija:

    $$
        (f(x)-g(x))' = f'(x) - g'(x)
    $$
\end{propositionbox}

Dokaz je identičan pravilu o derivaciji zbroja funkcija:

\begin{align*}
    &(f(x)-g(x))'=\\
    &=\lim_{h\to 0}\frac{f(x+h)-g(x+h)-(f(x)+g(x))}{h}\\
    &=\lim_{h\to 0}\frac{f(x+h)-f(x)-(g(x+h)-g(x))}{h}\\
    &=\lim_{h\to 0}\frac{f(x+h)-f(x)}{h} - \lim_{h\to 0}\frac{g(x+h)-g(x)}{h}\\
    &=f'(x) - g'(x)
\end{align*}

\newcolumn

\begin{propositionbox}[pravilo o deriviranju konstante]
    Derivacija konstante $c$ je jednaka nuli:

    $$
        \frac{d}{dx}(c) = 0
    $$
\end{propositionbox}

Zamjenimo konstantu $c$ sa funkcijom $f$ koja za svaki $x$ vraća vrijednost $c$, $f(x)=c$.
Iz toga slijedi:

\begin{align*}
    f'(x) =& \lim_{h\to 0}\frac{f(x+h)-f(x)}{h}\\
          =& \lim_{h\to 0}\frac{c - c}{h} = \lim_{h\to 0}0 = 0
\end{align*}

\begin{propositionbox}[pravilo o deriviranju umnoška funkcije s konstantom]
    Derivacija umnoška konstante i funkcije je jednaka umnošku te konstante i derivacije funkcije:
    $$
        (cf(x))' = cf'(x)
    $$
\end{propositionbox}

\begin{align*}
    (cf(x))'=&\lim_{h\to 0}\frac{cf(x+h)-cf(x)}{h}\\
            =&c\lim_{h\to 0}\frac{f(x+h)-f(x)}{h}\\
            =&cf'(x)
\end{align*}

\end{multicols}

\begin{propositionbox}[pravilo o deriviranju umnoška funkcija]
    \label{eq:diff_prod}
    $$
        (fg)'(x) = (f(x)\cdot g(x))' = f'(x)g(x) + f(x)g'(x)
    $$
\end{propositionbox}

\begin{align*}
    (fg)'(x)=&\lim_{h\to 0}\frac{cf(x+h)-cf(x)}{h}\\
    =&c\lim_{h\to 0}\frac{f(x+h)-f(x)}{h}\\
    =&cf'(x)
\end{align*}

\begin{propositionbox}[pravilo o deriviranju kvocijenta funkcija]
    $$
        \left(\frac{f(x)}{g(x)}\right)' = \frac{f'(x)g(x) - f(x)g'(x)}{(g(x))^2}
    $$
\end{propositionbox}

Ako je $h(x) = \frac{f(x)}{g(x)}$, onda vrijedi:

\begin{align*}
    h'(x) =& \lim_{k\to 0} \frac{h(x+k)-h(x)}{k}\\
    =& \lim_{k\to 0} \frac{\frac{f(x+k)}{g(x+k)}-\frac{f(x)}{g(x)}}{k}\\
    =& \lim_{k\to 0} \frac{f(x+k)g(x)-f(x)g(x+k)}{k\cdot g(x)g(x+k)}\\
    =& \lim_{k\to 0} \frac{f(x+k)g(x)-f(x)g(x+k)}{k} \cdot \lim_{k\to 0}\frac{1}{g(x)g(x+k)}\\
    =& \lim_{k\to 0} \left[\frac{f(x+k)g(x) - f(x)g(x) + f(x)g(x) - f(x)g(x+k)}{k} \right] \cdot \frac{1}{g(x)^2} \\
    =& \left[\lim_{k\to 0} \frac{f(x+k)g(x) - f(x)g(x)}{k} - \lim_{k\to 0}\frac{f(x)g(x+k) - f(x)g(x)}{k} \right] \cdot \frac{1}{g(x)^2} \\
    =& \left[\lim_{k\to 0} \frac{f(x+k) - f(x)}{k} \cdot g(x) - f(x) \cdot \lim_{k\to 0}\frac{g(x+k) - g(x)}{k} \right] \cdot \frac{1}{g(x)^2} \\
    =& \frac{f'(x)g(x) - f(x)g'(x)}{g(x)^2}.
\end{align*}

Gdje, jer je funkcija $g(x)$ neprekidna na promatranom intervalu, vrijedi:
$$
\lim_{k\to 0}\frac{1}{g(x)g(x+k)} = \lim_{k\to 0}\frac{1}{g(x)^2}
$$

\begin{propositionbox}[pravilo o deriviranju inverzne funkcije]
    $$
        (f^{-1})'(x) = \frac{1}{f'(f^{-1}(x))}
    $$
\end{propositionbox}


\begin{align*}
    f(f^{-1}(x)) = x \xrightarrow{der.}\qquad \frac{d}{dx} f(f^{-1}(x)) =& \frac{d}{dx} x \phantom{f(f^{-1}(x)) = x \to\qquad}\\
    \frac{d\left( f(f^{-1}(x)) \right)}{d\left( f^{-1}(x) \right)} \frac{d\left(f^{-1}(x)\right)}{dx}
    =& 1 \\
    \frac{df(f^{-1}(x))}{df^{-1}(x)} \frac{df^{-1}(x)}{dx}
    =& 1 \\
    f'(f^{-1}(x))
    (f^{-1})'(x)
    =& 1\\
    (f^{-1})'(x)
    =& \frac{1}{f'(f^{-1}(x))}\\
\end{align*}

\begin{propositionbox}[pravilo o deriviranju kompozicije funkcija]
    $$
        (f\circ g)'(x) = (f(g(x)))' = f'(g(x)) \cdot g'(x)
    $$
\end{propositionbox}

\begin{propositionbox}[pravilo o deriviranju eksponenta prirodnog broja]
    $$
        (e^x)' = e^x \cdot x'
    $$
\end{propositionbox}

\begin{propositionbox}[pravilo o deriviranju funkcije s potencijom]
    $$
        (x^{n})' = nx^{n-1}
    $$
\end{propositionbox}

\begin{propositionbox}[pravilo o deriviranju logaritamske funkcije]
    $$
        (\ln(u))' = \frac{u'}{u}
    $$
\end{propositionbox}

\subsection{Pravila deriviranja trigonometrijskih funkcija}

\begin{propositionbox}[pravilo o deriviranju sinusa]
    $$
        (\sin(x))' = \cos(x)
    $$
\end{propositionbox}

\begin{propositionbox}[pravilo o deriviranju kosinusa]
    $$
        (\cos(x))' = -\sin(x)
    $$
\end{propositionbox}

\begin{propositionbox}[pravilo o deriviranju tangensa]
    $$
        (\tan(x))' = \sec^2(x)
    $$
\end{propositionbox}

\begin{propositionbox}[pravilo o deriviranju kotangense]
    $$
        (\cot(x))' = -\csc^2(x)
    $$
\end{propositionbox}

\begin{propositionbox}[pravilo o deriviranju sekansa]
    $$
        (\sec(x))' = \sec(x) \tan(x)
    $$
\end{propositionbox}

\begin{propositionbox}[pravilo o deriviranju kosekansa]
    $$
        (\csc(x))' = -\csc(x) \cot(x)
    $$
\end{propositionbox}
