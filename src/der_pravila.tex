\subsection{Pravila deriviranja}

\begin{multicols}{2}

\begin{proposition}[pravilo o zbroju funkcija]
    Derivacija zbroja je jednaka zbroju derivacija:

    $$
        (f(x)+g(x))' = f'(x) + g'(x)
    $$
\end{proposition}

Raspisujemo zbroj funkcija kroz \fancyeq[definiciju za][derivaciju funkcije]{def:diff},
uređujemo nazivnik i primjenjujemo pravilo o zbroju limesa:

\begin{align*}
    &(f(x)+g(x))'=\\
    &=\lim_{h\to 0}\frac{f(x+h)+g(x+h)-(f(x)+g(x))}{h}\\
    &=\lim_{h\to 0}\frac{f(x+h)-f(x)+g(x+h)-g(x)}{h}\\
    &=\lim_{h\to 0}\frac{f(x+h)-f(x)}{h} + \lim_{h\to 0}\frac{g(x+h)-g(x)}{h}\\
    &=f'(x) + g'(x)
\end{align*}

\begin{proposition}[pravilo o razlici funkcija]
    Derivacija razlike je jednaka razlici derivacija:

    $$
        (f(x)-g(x))' = f'(x) - g'(x)
    $$
\end{proposition}

Dokaz je identičan pravilu o derivaciji zbroja funkcija:

\begin{align*}
    &(f(x)-g(x))'=\\
    &=\lim_{h\to 0}\frac{f(x+h)-g(x+h)-(f(x)+g(x))}{h}\\
    &=\lim_{h\to 0}\frac{f(x+h)-f(x)-(g(x+h)-g(x))}{h}\\
    &=\lim_{h\to 0}\frac{f(x+h)-f(x)}{h} - \lim_{h\to 0}\frac{g(x+h)-g(x)}{h}\\
    &=f'(x) - g'(x)
\end{align*}

\begin{proposition}[pravilo o deriviranju konstante]
    Derivacija konstante $c$ je jednaka nuli:

    $$
        \frac{d}{dx}(c) = 0
    $$
\end{proposition}

Zamjenimo konstantu $c$ sa funkcijom $f$ koja za svaki $x$ vraća vrijednost $c$, $f(x)=c$.
Iz toga slijedi:

\begin{align*}
    f'(x) =& \lim_{h\to 0}\frac{f(x+h)-f(x)}{h}\\
          =& \lim_{h\to 0}\frac{c - c}{h} = \lim_{h\to 0}0 = 0
\end{align*}

\vspace*{\stretch{1}}

\newcolumn

\begin{proposition}[pravilo o deriviranju umnoška funkcije s konstantom]
    Derivacija umnoška konstante i funkcije je jednaka umnošku te konstante i derivacije funkcije:
    $$
        (cf(x))' = cf'(x)
    $$
\end{proposition}

\begin{align*}
    (cf(x))'=&\lim_{h\to 0}\frac{cf(x+h)-cf(x)}{h}\\
            =&c\lim_{h\to 0}\frac{f(x+h)-f(x)}{h}\\
            =&cf'(x)
\end{align*}

\begin{proposition}[pravilo o deriviranju umnoška funkcija]
    \label{eq:diff_prod}
    $$
        (fg)'(x) = (f(x)\cdot g(x))' = f'(x)g(x) + f(x)g'(x)
    $$
\end{proposition}

\begin{align*}
    (fg)'(x)=&\lim_{h\to 0}\frac{cf(x+h)-cf(x)}{h}\\
    =&c\lim_{h\to 0}\frac{f(x+h)-f(x)}{h}\\
    =&cf'(x)
\end{align*}

\begin{proposition}[pravilo o deriviranju inverzne funkcije]
    $$
        (f^{-1})'(x) = \frac{1}{f'(f^{-1}(x))}
    $$
\end{proposition}

\begin{align*}
    f(f^{-1}(x)) = x \xrightarrow{der.}\qquad \frac{d}{dx} f(f^{-1}(x)) =& \frac{d}{dx} x \phantom{f(f^{-1}(x)) = x \to\qquad}\\
    \frac{d\left( f(f^{-1}(x)) \right)}{d\left( f^{-1}(x) \right)} \frac{d\left(f^{-1}(x)\right)}{dx}
    =& 1 \\
    \frac{df(f^{-1}(x))}{df^{-1}(x)} \frac{df^{-1}(x)}{dx}
    =& 1 \\
    f'(f^{-1}(x))
    (f^{-1})'(x)
    =& 1\\
    (f^{-1})'(x)
    =& \frac{1}{f'(f^{-1}(x))}\\
\end{align*}

\vspace*{\stretch{1}}

\end{multicols}

\pagebreak

\begin{proposition}[pravilo o deriviranju kvocijenta funkcija]
    $$
        \left(\frac{f(x)}{g(x)}\right)' = \frac{f'(x)g(x) - f(x)g'(x)}{(g(x))^2}
    $$
\end{proposition}

\begin{align*}
    h'(x) =& \lim_{h\to 0} \frac{\dfrac{f(x+h)}{g(x+h)}-\dfrac{f(x)}{g(x)}}{h} = \lim_{h\to 0} \frac{f(x+h)g(x)-f(x)g(x+h)}{h\cdot g(x)g(x+h)}\\
    =& \lim_{h\to 0} \frac{f(x+h)g(x)-f(x)g(x+h)}{h} \cdot \lim_{h\to 0}\frac{1}{g(x)g(x+h)}\\
    =& \lim_{h\to 0} \left[\frac{f(x+h)g(x) - f(x)g(x) + f(x)g(x) - f(x)g(x+h)}{h} \right] \cdot \frac{1}{g(x)^2} \\
    =& \left[\lim_{h\to 0} \frac{f(x+h)g(x) - f(x)g(x)}{h} - \lim_{h\to 0}\frac{f(x)g(x+h) - f(x)g(x)}{h} \right] \cdot \frac{1}{g(x)^2} \\
    =& \left[\lim_{h\to 0} \frac{f(x+h) - f(x)}{h} \cdot g(x) - f(x) \cdot \lim_{h\to 0}\frac{g(x+h) - g(x)}{h} \right] \cdot \frac{1}{g(x)^2} \\
    =& \frac{f'(x)g(x) - f(x)g'(x)}{g(x)^2}.
\end{align*}

Gdje, jer je funkcija $g(x)$ neprekidna na promatranom intervalu, vrijedi:
$
\displaystyle
\lim_{k\to 0}\frac{1}{g(x)g(x+k)} = \lim_{k\to 0}\frac{1}{g(x)^2}
$

\begin{proposition}[pravilo o deriviranju kompozicije funkcija]
    $$
        (f\circ g)'(x) = (f(g(x)))' = f'(g(x)) \cdot g'(x)
    $$
\end{proposition}

\begin{align*}
    (f \circ g)'(x) =& \lim_{h \to 0} \frac{(f \circ g)(x+h) - (f \circ g)(x)}{h} = \lim_{h \to 0} \frac{g(f(x+h))-g(f(x))}{h}\\
    =& \lim_{h \to 0} \frac{g(f(x+h))-g(f(x))}{h} \cdot \frac{f(x+h)-f(x)}{f(x+h)-f(x)} = \lim_{h \to 0} \frac{g(f(x+h))-g(f(x))}{f(x+h)-f(x)} \cdot \frac{f(x+h)-f(x)}{h}\\
    =& \lim_{h \to 0} \frac{g(f(x+h))-g(f(x))}{f(x+h)-f(x)} \cdot \lim_{h \to 0} \frac{f(x+h)-f(x)}{h}\\
    =& \lim_{h \to 0} \frac{g(f(x+h))-g(f(x))}{f(x+h)-f(x)} \cdot f'(x)\\
\end{align*}

Koristimo promjenu u varijabli: $k = f(x+h) - f(x) \implies f(x+h) = f(x) + k$,
iz čega dobivamo:
$$
\lim_{h\to 0} k = \lim_{h\to 0} f(x+h) - f(x) = f(x+0) - f(x) = 0
$$

Što znači da kada se $h$ približava $0$, tada se i $k$ približava $0$. Traženje
limesa kada se $h$ približava $0$, korištenjem promjene u varijabli, je
identično traženju limesa kada se $k$ približava $0$:
\begin{align*}
    \lim_{h\to 0} \frac{g(f(x+h)) - g(f(x))}{f(x+h) - f(x)}
    =& \lim_{k\to 0} \frac{g(f(x) + k) - g(f(x))}{k}\\
    =& g'(f(x))
\end{align*}

Time dobivamo:
\begin{align*}
    (f \circ g)'(x) =& \lim_{h \to 0} \frac{g(f(x+h))-g(f(x))}{f(x+h)-f(x)} \cdot f'(x)\\
    =& \lim_{k\to 0} \frac{g(f(x) + k) - g(f(x))}{k} \cdot f'(x)\\
    =& g'(f(x)) \cdot f'(x)
\end{align*}

\pagebreak

\begin{proposition}[pravilo o deriviranju eksponenta prirodnog broja]
    $$
        (e^x)' = e^x\quad \text{ili} \quad \exp'(x) = \exp(x)
    $$
\end{proposition}

\begin{multicols}{2}
\begin{align*}
    (e^x)' =& \lim_{h\to 0 } \frac{e^{x+h} - e^x}{h}\\
    =& \lim_{h\to 0} \frac{e^x \cdot e^h - e^x}{h}\\
    =& \lim_{h\to 0} \frac{e^x (e^h - 1)}{h}\\
    =& e^x \lim_{h\to 0} \frac{e^h - 1}{h} = e^x
\end{align*}

\vspace*{\stretch{1}}
\columnbreak

\noindent
Član $\lim_{h\to 0} \frac{e^h - 1}{h}$ iz predzadnjeg koraka je jednak $1$.

To se može najjednostavnije dokazati primjenom L'Hôpitalovog pravila koje je
kasnije objašnjeno:
\begin{align*}
    \lim_{h\to 0} \frac{e^h - 1}{h} = \lim_{h\to 0} \frac{(e^h - 1)'}{(h)'} =& \lim_{h\to 0} \frac{e^h}{1}\\
    =& e^0 = 1\,,
\end{align*}
no taj dokaz ovisi o pravilu o deriviranju eksponenta prirodnog broja.

\end{multicols}

Dokaz da je $\lim_{h\to 0} \frac{e^h - 1}{h} = 1$ koji ne ovisi o pravilu o
deriviranju eksponenta prirodnog broja koristi \textit{definiciju eksponenta kao
limesa niza} je:
\begin{align*}
    \frac{e^h - 1}{h} =& \frac{\displaystyle \lim_{n\to \infty} \left( 1 + \dfrac{h}{n} \right)^n - 1}{h}
    = \frac{\displaystyle \lim_{n \to \infty} \sum_{k=0}^{n} \binom{n}{k} \left(\frac{h}{n}\right)^k - 1}{h}
    = \lim_{n \to \infty} \frac{\displaystyle \sum_{k=0}^{n} \binom{n}{k} \left(\dfrac{h}{n}\right)^k - 1}{h}\\
    =& \lim_{n \to \infty} \left(\binom{n}{0} 1 - 1 + \binom{n}{1}\left(\frac{h}{n}\right)\frac{1}{h} + \sum_{k=2}^n \binom{n}{k}\left(\frac{h}{n}\right)^k\frac{1}{h} \right)\\
    =& \lim_{n \to \infty} 1 + \lim_{n \to \infty} \sum_{k=2}^n \binom{n}{k} \frac{h^{k-1}}{n^k} = 1 + h \lim_{n \to \infty} \sum_{k=2}^n \binom{n}{k} \frac{h^{k-1}}{n^k}\,.
\end{align*}
Kako se $h$ približava $0$, drugi pribrojnik se isto približava $0$, pa vrijedi:
$
\displaystyle \lim_{h\to 0} \frac{e^h - 1}{h} = 1
$

\begin{multicols}{2}

\begin{proposition}[pravilo o deriviranju logaritamske funkcije]
    $$
        (\ln x)' = \frac{1}{x}
    $$
\end{proposition}

Koristimo činjenicu da je eksponent prirodnog broja inverz prirodnom logaritmu: $y = e^x \Longleftrightarrow x = \ln y$.

Uvrstimo li dokazano pravilo o deriviranju eksponenta prirodnog broja $(e^x)' = e^x$ u pravilo o deriviranju inverzne funkcije $(f^{-1})'(x) = \dfrac{1}{f'(f^{-1}(x))}$, dobivamo:
\begin{align*}
    (\ln x)' = \frac{1}{e^{\ln x}} = \frac{1}{x}
\end{align*}

\columnbreak

\begin{proposition}[pravilo o deriviranju eksponenta funkcije]
    $$
        (a^x)' = a^x \cdot \ln a,\quad a > 0
    $$
\end{proposition}

Jer je $a^x = e^{x \ln a},\, \forall a \in \mathbb{R}_{>0}$, vrijedi:

\begin{align*}
    (a^x)' = (e^{x \ln a})' = e^{x \ln a} \cdot \ln a = a^x \cdot \ln a
\end{align*}

\end{multicols}

\pagebreak

\begin{proposition}[pravilo o deriviranju funkcije s potencijom]
    $$
        (x^{n})' = nx^{n-1}\quad \text{ili} \quad (f(x)^{n})' = n\cdot f(x)^{n-1}
    $$
\end{proposition}

\begin{align*}
    (f(x)^{n})' =& \lim_{h\to 0} \frac{f(x+h)^n - (f(x))^n}{h}\\
    =& (f(x))^n \lim_{h\to 0} \dfrac{\left(\dfrac{f(x+h)}{f(x)}\right)^n - 1}{h}
    = (f(x))^n \lim_{h\to 0} \dfrac{e^{n \ln \frac{f(x+h)}{f(x)}} - 1}{h}\\
    =& (f(x))^n \lim_{h\to 0} \left(\dfrac{e^{n \ln \frac{f(x+h)}{f(x)}} - 1}{n \ln \frac{f(x+h)}{f(x)}}\right)\left(\frac{n \ln \frac{f(x+h)}{f(x)}}{h}\right)\\
    =& (f(x))^n \lim_{h\to 0} \frac{n \ln \frac{f(x+h)}{f(x)}}{h}
    = n(f(x))^n \lim_{h\to 0} \frac{\ln \frac{f(x+h)}{f(x)}}{h}\\
    =& n(f(x))^n \lim_{h\to 0} \frac{\ln \left(1 + \frac{f(x+h) - f(x)}{f(x)} \right)}{h}\\
    =& n(f(x))^n \lim_{h\to 0} \left(\frac{\ln \left(1 + \frac{f(x+h) - f(x)}{f(x)} \right)}{\frac{f(x+h) - f(x)}{f(x)}} \right) \left( \frac{\frac{f(x+h) - f(x)}{f(x)}}{h} \right)\\
    =& n(f(x))^n \lim_{h\to 0} \frac{\left(\frac{f(x+h)-f(x)}{f(x)}\right)}{h}
    = n(f(x))^n \lim_{h\to 0} \frac{1}{f(x)}\frac{f(x+h)-f(x)}{h}\\
    =& n(f(x))^{n-1} \lim_{h\to 0} \frac{f(x+h)-f(x)}{h} = n(f(x))^{n-1} f'(x)
\end{align*}

\subsection{Pravila deriviranja trigonometrijskih funkcija}

\begin{multicols}{2}
\begin{proposition}[pravilo o deriviranju sinusa]
    $$
        (\sin(x))' = \cos(x)
    $$
\end{proposition}

\begin{proposition}[pravilo o deriviranju kosinusa]
    $$
        (\cos(x))' = -\sin(x)
    $$
\end{proposition}

\begin{proposition}[pravilo o deriviranju tangensa]
    $$
        (\tan(x))' = \sec^2(x)
    $$
\end{proposition}

\begin{proposition}[pravilo o deriviranju kotangensa]
    $$
        (\cot(x))' = -\csc^2(x)
    $$
\end{proposition}

\begin{proposition}[pravilo o deriviranju sekansa]
    $$
        (\sec(x))' = \sec(x) \tan(x)
    $$
\end{proposition}

\begin{proposition}[pravilo o deriviranju kosekansa]
    $$
        (\csc(x))' = -\csc(x) \cot(x)
    $$
\end{proposition}

\end{multicols}
